\chapter{The Unix Shell}\label{s:shell}

The Unix shell has been around longer than most of its users have been
alive. It has survived so long because it's a power tool that allows
people to do complex things with just a few keystrokes. More
importantly, it helps them combine existing programs in new ways and
automate repetitive tasks so that they don't have to type the same
things over and over again.

\section{Introducing the Shell}

\begin{objectives}
\begin{swcitemize}
\item
  Explain how the shell relates to the keyboard, the screen, the
  operating system, and users' programs.
\item
  Explain when and why command-line interfaces should be used instead of
  graphical interfaces.
\end{swcitemize}
\end{objectives}

Nelle Nemo, a marine biologist, has just returned from a six-month
survey of the
\urlfoot{http://en.wikipedia.org/wiki/North\_Pacific\_Gyre}{North Pacific
Gyre}, where she has been sampling gelatinous marine life in the
\urlfoot{http://en.wikipedia.org/wiki/Great\_Pacific\_Garbage\_Patch}{Great
Pacific Garbage Patch}. She has 300 samples in all, and now needs to:

\begin{swcenumerate}
\item
  Run each sample through an assay machine that will measure the
  relative abundance of 300 different proteins. The machine's output for
  a single sample is a file with one line for each protein.
\item
  Calculate statistics for each of the proteins separately using a
  program her supervisor wrote called \code{goostat}.
\item
  Compare the statistics for each protein with corresponding statistics
  for each other protein using a program one of the other graduate
  students wrote called \code{goodiff}.
\item
  Write up. Her supervisor would really like her to do this by the end
  of the month so that her paper can appear in an upcoming special issue
  of \emph{Aquatic Goo Letters}.
\end{swcenumerate}

It takes about half an hour for the assay machine to process each
sample. The good news is, it only takes two minutes to set each one up.
Since her lab has eight assay machines that she can use in parallel,
this step will ``only'' take about two weeks.

The bad news is that if she has to run \code{goostat} and
\code{goodiff} by hand, she'll have to enter filenames and click
``OK'' 45,150 times (300 runs of \code{goostat}, plus 300${\times}$299/2 runs
of \code{goodiff}). At 30 seconds each, that will take more than two
weeks. Not only would she miss her paper deadline, the chances of her
typing all of those commands right are practically zero.

This chapter will explore what she should do instead. More
specifically, it will explain how she can use a command shell to
automate the repetitive steps in her processing pipeline so that her
computer can work 24 hours a day while she writes her paper. As a
bonus, once she has put a processing pipeline together, she will be
able to use it again whenever she collects more data.

\subsection*{What and Why}

At a high level, computers do four things:

\begin{swcitemize}
\item
  run programs;
\item
  store data;
\item
  communicate with each other; and
\item
  interact with us.
\end{swcitemize}

They can do the last of these in many different ways, including direct
brain-computer links and speech interfaces. Since these are still in
their infancy, most of us use windows, icons, mice, and pointers. These
technologies didn't become widespread until the 1980s, but their roots
go back to Doug Engelbart's work in the 1960s, which you can see in what
has been called ``\urlfoot{http://www.youtube.com/watch?v=a11JDLBXtPQ}{The
Mother of All Demos}''.

Going back even further, the only way to interact with early computers
was to rewire them. But in between, from the 1950s to the 1980s, most
people used line printers. These devices only allowed input and output
of the letters, numbers, and punctuation found on a standard keyboard,
so programming languages and interfaces had to be designed around that
constraint.

This kind of interface is called a \gl{command-line
interface}{g:cli}, or CLI, to distinguish it from the
\gl{graphical user interface}{g:gui}, or GUI, that most people now
use. The heart of a CLI is a \gl{read-evaluate-print
loop}{g:repl}, or REPL: when the user types a command and then presses the enter
(or return) key, the computer reads it, executes it, and prints its
output. The user then types another command, and so on until the user
logs off.

This description makes it sound as though the user sends commands
directly to the computer, and the computer sends output directly to the
user. In fact, there is usually a program in between called a
\gl{command shell}{g:shell}. What the user types goes into the
shell; it figures out what commands to run and orders the computer to
execute them.

A shell is a program like any other. What's special about it is that its
job is to run other programs rather than to do calculations itself. The
most popular Unix shell is Bash, the Bourne Again SHell (so-called
because it's derived from a shell written by Stephen Bourne---this is
what passes for wit among programmers). Bash is the default shell on
most modern implementations of Unix, and in most packages that provide
Unix-like tools for Windows.

Using Bash or any other shell sometimes feels more like programming than
like using a mouse. Commands are terse (often only a couple of
characters long), their names are frequently cryptic, and their output
is lines of text rather than something visual like a graph. On the other
hand, the shell allows us to combine existing tools in powerful ways
with only a few keystrokes and to set up pipelines to handle large
volumes of data automatically. In addition, the command line is often
the easiest way to interact with remote machines. As clusters and cloud
computing become more popular for scientific data crunching, being able
to drive them is becoming a necessary skill.

\begin{keypoints}
\begin{swcitemize}
\item
  A shell is a program whose primary purpose is to read commands and run
  other programs.
\item
  The shell's main advantages are its high action-to-keystroke ratio,
  its support for automating repetitive tasks, and that it can be used
  to access networked machines.
\item
  The shell's main disadvantages are its primarily textual nature and
  how cryptic its commands and operation can be.
\end{swcitemize}
\end{keypoints}

\section{Files and Directories}

\begin{objectives}
\begin{swcitemize}
\item
  Explain the similarities and differences between a file and a
  directory.
\item
  Translate an absolute path into a relative path and vice versa.
\item
  Construct absolute and relative paths that identify specific files and
  directories.
\item
  Explain the steps in the shell's read-run-print cycle.
\item
  Identify the actual command, flags, and filenames in a command-line
  call.
\item
  Demonstrate the use of tab completion, and explain its advantages.
\end{swcitemize}
\end{objectives}

The part of the operating system responsible for managing files and
directories is called the \gl{file system}{g:filesystem}. It
organizes our data into files, which hold information, and directories
(also called ``folders''), which hold files or other directories.

Several commands are frequently used to create, inspect, rename, and
delete files and directories. To start exploring them, let's open a
shell window:

\begin{VerbIn}
$
\end{VerbIn}

The dollar sign is a \gl{prompt}{g:prompt}, which shows us that
the shell is waiting for input; your shell may show something more
elaborate.

Type the command \code{whoami}, then press the Enter key (sometimes
marked Return) to send the command to the shell. The command's output is
the ID of the current user, i.e., it shows us who the shell thinks we
are:

\begin{VerbIn}
$ whoami
\end{VerbIn}
% [c]
\begin{VerbOut}
nelle
\end{VerbOut}

\noindent
More specifically, when we type \code{whoami} the shell:

\begin{swcenumerate}
\item
  finds a program called \code{whoami},
\item
  runs that program,
\item
  displays that program's output, then
\item
  displays a new prompt to tell us that it's ready for more commands.
\end{swcenumerate}

Next, let's find out where we are by running a command called
\code{pwd} (which stands for ``print working directory''). At any
moment, our \gl{current working
directory}{g:current-working-directory} is our current default directory, i.e., the directory that
the computer assumes we want to run commands in unless we explicitly
specify something else. Here, the computer's response is
\code{/users/nelle}, which is Nelle's \gl{home
directory}{g:home-directory}:

\begin{VerbIn}
$ pwd
\end{VerbIn}

% [c]
\begin{VerbOut}
/users/nelle
\end{VerbOut}

\begin{swcbox}{Alphabet Soup}

If the command to find out who we are is \code{whoami}, why is the
command to find out where we are called \code{pwd} instead of
\code{whereami}?  The usual answer is that in the early 1970s, when
Unix was first being developed, every keystroke counted: the devices
of the day were slow, and backspacing on a teletype was so painful
that cutting the number of keystrokes in order to cut the number of
typing mistakes was actually a win for usability. The reality is that
commands were added to Unix one by one, without any master plan, by
people who were immersed in its jargon. The result is as inconsistent
as the roolz uv Inglish speling, but we're stuck with it now.

\end{swcbox}

To understand what a ``home directory'' is, let's have a look at how the
file system as a whole is organized. At the top is the
\gl{root directory}{g:root-directory} that holds everything else.
We refer to it using a slash character \code{/} on its own; this is
the leading slash in \code{/users/nelle}.

Inside that directory are several other directories: \code{bin} (which
is where some built-in programs are stored), \code{data} (for
miscellaneous data files), \code{users} (where users' personal
directories are located), \code{tmp} (for temporary files that don't
need to be stored long-term), and so on (\figref{f:filesystem}).

\swcgraphics{f:filesystem}{A Typical Filesystem}{novice/shell/img/filesystem.pdf}{0.75}

We know that our current working directory \code{/users/nelle} is
stored inside \code{/users} because \code{/users} is the first part
of its name. Similarly, we know that \code{/users} is stored inside
the root directory \code{/} because its name begins with \code{/}.

Underneath \code{/users}, we find one directory for each user with
an account on this machine (\figref{f:homedir}). The Mummy's files are
stored in \code{/users/imhotep}, Wolfman's in \code{/users/larry},
and ours in \code{/users/nelle}, which is why \code{nelle} is the
last part of the directory's name.

\swcgraphics{f:homedir}{Home Directories}{novice/shell/img/home-directories.pdf}{0.75}

\begin{quote} % FIXME
Notice that there are two meanings for the \code{/} character. When it
appears at the front of a file or directory name, it refers to the root
directory. When it appears \emph{inside} a name, it's just a separator.
\end{quote}

Let's see what's in Nelle's home directory by running \code{ls}, which
stands for ``listing'' (\figref{f:listing}):

\begin{VerbIn}
$ ls
\end{VerbIn}

% [c]
\begin{VerbOut}
bin          data      mail       music
notes.txt    papers    pizza.cfg  solar
solar.pdf    swc
\end{VerbOut}

\swcgraphics{f:listing}{Nelle's Home Directory}{novice/shell/img/homedir.pdf}{0.75}

\code{ls} prints the names of the files and directories in the current
directory in alphabetical order, arranged neatly into columns. We can
make its output more comprehensible by using the
\gl{flag}{g:command-line-flag} \code{-F}, which tells
\code{ls} to add a trailing \code{/} to the names of directories:

\begin{VerbIn}
$ ls -F
\end{VerbIn}

% [c]
\begin{VerbOut}
bin/         data/     mail/      music/
notes.txt    papers/   pizza.cfg  solar/
solar.pdf    swc/
\end{VerbOut}

Here, we can see that \code{/users/nelle} contains seven
\gl{sub-directories}{g:sub-directory}. The names that don't have
trailing slashes, like \code{notes.txt}, \code{pizza.cfg}, and
\code{solar.pdf}, are plain old files. And note that there is a space
between \code{ls} and \code{-F}: without it, the shell thinks we're
trying to run a command called \code{ls-F}, which doesn't exist.

\begin{swcbox}{What's In A Name?}

You may have noticed that all of Nelle's files' names are ``something dot
something''. This is just a convention: we can call a file
\code{mythesis} or almost anything else we want. However, most people
use two-part names most of the time to help them (and their programs)
tell different kinds of files apart. The second part of such a name is
called the \gl{filename extension}{g:filename-extension}, and
indicates what type of data the file holds: \code{.txt} signals a
plain text file, \code{.pdf} indicates a PDF document, \code{.cfg}
is a configuration file full of parameters for some program or other,
and so on.

This is just a convention, albeit an important one. Files contain
bytes: it's up to us and our programs to interpret those bytes
according to the rules for PDF documents, images, and so on.  Naming a
PNG image of a whale as \code{whale.mp3} doesn't somehow magically
turn it into a recording of whalesong, though it \emph{might} cause
the operating system to try to open it with a music player when
someone double-clicks it.

\end{swcbox}

Now let's take a look at what's in Nelle's \code{data} directory by
running \code{ls -F data}, i.e., the command \code{ls} with the
arguments \code{-F} and \code{data}. The second argument---the one
\emph{without} a leading dash---tells \code{ls} that we want a listing
of something other than our current working directory:

\begin{VerbIn}
$ ls -F data
\end{VerbIn}

% [c]
\begin{VerbOut}
amino-acids.txt   elements/     morse.txt
pdb/              planets.txt   sunspot.txt
\end{VerbOut}

The output shows us that there are four text files and two
sub-sub-directories. Organizing things hierarchically in this way helps
us keep track of our work: it's possible to put hundreds of files in our
home directory, just as it's possible to pile hundreds of printed papers
on our desk, but it's a self-defeating strategy.

Notice, by the way that we spelled the directory name \code{data}. It
doesn't have a trailing slash: that's added to directory names by
\code{ls} when we use the \code{-F} flag to help us tell things
apart. And it doesn't begin with a slash because it's a
\gl{relative path}{g:relative-path}, i.e., it tells \code{ls}
how to find something from where we are, rather than from the root of
the file system.

If we run \code{ls -F /data} (\emph{with} a leading slash) we get a
different answer, because \code{/data} is an
\gl{absolute path}{g:absolute-path}:

\begin{VerbIn}
$ ls -F /data
\end{VerbIn}

% [c]
\begin{VerbOut}
access.log    backup/    hardware.cfg
network.cfg
\end{VerbOut}

\noindent
The leading \code{/} tells the computer to follow the path from the
root of the filesystem, so it always refers to exactly one directory, no
matter where we are when we run the command.

What if we want to change our current working directory? Before we do
this, \code{pwd} shows us that we're in \code{/users/nelle}, and
\code{ls} without any arguments shows us that directory's contents:

\begin{VerbIn}
$ pwd
\end{VerbIn}

% [c]
\begin{VerbOut}
/users/nelle
\end{VerbOut}

% [c]
\begin{VerbIn}
$ ls
\end{VerbIn}

% [c]
\begin{VerbOut}
bin/         data/     mail/      music/
notes.txt    papers/   pizza.cfg  solar/
solar.pdf    swc/
\end{VerbOut}

We can use \code{cd} followed by a directory name to change our
working directory. \code{cd} stands for ``change directory'', which is
a bit misleading: the command doesn't change the directory, it changes
the shell's idea of what directory we are in.

\begin{VerbIn}
$ cd data
\end{VerbIn}

\code{cd} doesn't print anything, but if we run \code{pwd} after it,
we can see that we are now in \code{/users/nelle/data}. If we run
\code{ls} without arguments now, it lists the contents of
\code{/users/nelle/data}, because that's where we now are:

\begin{VerbIn}
$ pwd
\end{VerbIn}

% [c]
\begin{VerbOut}
/users/nelle/data
\end{VerbOut}

% [c]
\begin{VerbIn}
$ ls
\end{VerbIn}

% [c]
\begin{VerbOut}
amino-acids.txt   elements/     morse.txt
pdb/              planets.txt   sunspot.txt
\end{VerbOut}

We now know how to go down the directory tree: how do we go up? We could
use an absolute path:

\begin{VerbIn}
$ cd /users/nelle
\end{VerbIn}

\noindent
but it's almost always simpler to use \code{cd ..} to go up one level:

\begin{VerbIn}
$ pwd
\end{VerbIn}

% [c]
\begin{VerbOut}
/users/nelle/data
\end{VerbOut}

% [c]
\begin{VerbIn}
$ cd ..
$ pwd
\end{VerbIn}

% [c]
\begin{VerbOut}
/users/nelle
\end{VerbOut}

\code{..} is a special directory name meaning ``the directory
containing this one'', or more succinctly, the
\gl{parent}{g:parent-directory} of the current directory. Sure
enough, if we run \code{pwd} after running \code{cd ..}, we're back
in \code{/users/nelle}.

The special directory \code{..} doesn't usually show up when we run
\code{ls}. If we want to display it, we can give \code{ls} the
\code{-a} flag:

\begin{VerbIn}
$ ls -F -a
\end{VerbIn}

% [c]
\begin{VerbOut}
./           ../       bin/       data/
mail/        music/    notes.txt  papers/
pizza.cfg    solar/    solar.pdf    swc/
\end{VerbOut}

\noindent
\code{-a} stands for ``show all''; it forces \code{ls} to show us
file and directory names that begin with \code{.}, such as \code{..}
(which, if we're in \code{/users/nelle}, refers to the \code{/users}
directory). As you can see, it also displays another special directory
that's just called \code{.}, which means ``the current working
directory''. It may seem redundant to have a name for it, but we'll see
some uses for it soon.

\begin{swcbox}{Orthogonality}

The special names \code{.} and \code{..} don't belong to
\code{ls}; they are interpreted the same way by every program. For
example, if we are in \code{/users/nelle/data}, the command
\code{ls ..} will give us a listing of \code{/users/nelle}. When the
meanings of the parts are the same no matter how they're combined,
programmers say they are \gl{orthogonal}{g:orthogonal}: Orthogonal
systems tend to be easier for people to learn because there are fewer
special cases and exceptions to keep track of.

\end{swcbox}

\subsection*{Nelle's Pipeline: Organizing Files}

Knowing just this much about files and directories, Nelle is ready to
organize the files that the protein assay machine will create. First,
she creates a directory called \code{north-pacific-gyre} (to remind
herself where the data came from). Inside that, she creates a directory
called \code{2012-07-03}, which is the date she started processing the
samples. She used to use names like \code{conference-paper} and
\code{revised-results}, but she found them hard to understand after a
couple of years. (The final straw was when she found herself creating a
directory called \code{revised-revised-results-3}.)

\begin{quote}
Nelle names her directories ``year-month-day'', with leading zeroes for
months and days, because the shell displays file and directory names in
alphabetical order. If she used month names, December would come before
July; if she didn't use leading zeroes, November ('11') would come
before July ('7').
\end{quote}

Each of her physical samples is labelled according to her lab's
convention with a unique ten-character ID, such as ``NENE01729A''. This
is what she used in her collection log to record the location, time,
depth, and other characteristics of the sample, so she decides to use it
as part of each data file's name. Since the assay machine's output is
plain text, she will call her files \code{NENE01729A.txt},
\code{NENE01812A.txt}, and so on. All 1520 files will go into the same
directory.

If she is in her home directory, Nelle can see what files she has using
the command:

\begin{VerbIn}
$ ls north-pacific-gyre/2012-07-03/
\end{VerbIn}

\noindent
This is a lot to type, but she can let the shell do most of the work. If
she types:

\begin{VerbIn}
$ ls no
\end{VerbIn}

\noindent
and then presses tab, the shell automatically completes the directory
name for her:

\begin{VerbIn}
$ ls north-pacific-gyre/
\end{VerbIn}

If she presses tab again, Bash will add \code{2012-07-03/} to the
command, since it's the only possible completion. Pressing tab again
does nothing, since there are 1520 possibilities; pressing tab twice
brings up a list of all the files, and so on. This is called
\gl{tab completion}{g:tab-completion}, and we will see it in many
other tools as we go on.

\begin{keypoints}
\begin{swcitemize}
\item
  The file system is responsible for managing information on the disk.
\item
  Information is stored in files, which are stored in directories
  (folders).
\item
  Directories can also store other directories, which forms a directory
  tree.
\item
  \code{/} on its own is the root directory of the whole filesystem.
\item
  A relative path specifies a location starting from the current
  location.
\item
  An absolute path specifies a location from the root of the filesystem.
\item
  Directory names in a path are separated with `/' on Unix, but
  `\textbackslash{}' on Windows.
\item
  `..' means ``the directory above the current one'', and `.' on its own
  means ``the current directory''.
\item
  Most files' names are \code{something.extension}. The extension
  isn't required, and doesn't guarantee anything, but is normally used
  to indicate the type of data in the file.
\item
  Most commands take options (flags) which begin with a `-'.
\end{swcitemize}
\end{keypoints}

Please refer to \figref{f:filesys-challenge} for the following challenges.

\swcgraphics{f:filesys-challenge}{File System for Challenge Questions}{novice/shell/img/filesystem-challenge.pdf}{0.75}

\begin{challenge}

  If \code{pwd} displays \code{/users/thing}, what will
  \code{ls ../backup} display? 1.
  \code{../backup: No such file or directory} 2.
  \code{2012-12-01 2013-01-08 2013-01-27} 3.
  \code{2012-12-01/ 2013-01-08/ 2013-01-27/} 4.
  \code{original pnas\_final pnas\_sub}

\end{challenge}

\begin{challenge}

  If \code{pwd} displays \code{/users/backup}, and \code{-r} tells
  \code{ls} to display things in reverse order, what command will
  display:

\begin{VerbOut}
pnas-sub/ pnas-final/ original/
\end{VerbOut}

\begin{swcenumerate}
\item
  \code{ls pwd}
\item
  \code{ls -r -F}
\item
  \code{ls -r -F /users/backup}
\item
  Either \#2 or \#3 above, but not \#1.
\end{swcenumerate}

\end{challenge}

\begin{challenge}

  What does the command \code{cd} without a directory name do? 1. It has
  no effect. 2. It changes the working directory to \code{/}. 3. It
  changes the working directory to the user's home directory. 4. It
  produces an error message.

\end{challenge}

\begin{challenge}

  What does the command \code{ls} do when used with the -s and -h
  arguments?

\end{challenge}

\section{Creating Things}

\begin{objectives}
\begin{swcitemize}
\item
  Create a directory hierarchy that matches a given diagram.
\item
  Create files in that hierarchy using an editor or by copying and
  renaming existing files.
\item
  Display the contents of a directory using the command line.
\item
  Delete specified files and/or directories.
\end{swcitemize}
\end{objectives}

We now know how to explore files and directories, but how do we create
them in the first place? Let's go back to Nelle's home directory,
\code{/users/nelle}, and use \code{ls -F} to see what it contains:

\begin{VerbIn}
$ pwd
\end{VerbIn}

% [c]
\begin{VerbOut}
/users/nelle
\end{VerbOut}

% [c]
\begin{VerbIn}
$ ls -F
\end{VerbIn}

% [c]
\begin{VerbOut}
bin/         data/     mail/      music/
notes.txt    papers/   pizza.cfg  solar/
solar.pdf    swc/
\end{VerbOut}

Let's create a new directory called \code{thesis} using the command
\code{mkdir thesis} (which has no output):

\begin{VerbIn}
$ mkdir thesis
\end{VerbIn}

As you might (or might not) guess from its name, \code{mkdir} means
``make directory''. Since \code{thesis} is a relative path (i.e.,
doesn't have a leading slash), the new directory is made below the
current working directory:

\begin{VerbIn}
$ ls -F
\end{VerbIn}

% [c]
\begin{VerbOut}
bin/         data/     mail/      music/
notes.txt    papers/   pizza.cfg  solar/
solar.pdf    swc/      thesis/
\end{VerbOut}

However, there's nothing in it yet:

\begin{VerbIn}
$ ls -F thesis
\end{VerbIn}

Let's change our working directory to \code{thesis} using \code{cd},
then run a text editor called Nano to create a file called
\code{draft.txt} (\figref{f:nano}).

\begin{VerbIn}
$ cd thesis
$ nano draft.txt
\end{VerbIn}

\swcgraphics{f:nano}{The Nano Editor}{novice/shell/img/nano-screenshot.png}{0.35}

\begin{swcbox}{Which Editor?}

When we say, ``\code{nano} is a text editor'' we really do mean
``text'': it can only work with plain character data, not tables,
images, or any other human-friendly media. We use it in examples because
almost anyone can drive it anywhere without training, but please use
something more powerful for real work. On Unix systems (such as Linux
and Mac OS X), many programmers use
\urlfoot{http://www.gnu.org/software/emacs/}{Emacs} or
\urlfoot{http://www.vim.org/}{Vim} (both of which are completely
unintuitive, even by Unix standards), or a graphical editor such as
\urlfoot{http://projects.gnome.org/gedit/}{Gedit}. On Windows, you may wish
to use \urlfoot{http://notepad-plus-plus.org/}{Notepad++}.

No matter what editor you use, you will need to know where it searches
for and saves files. If you start it from the shell, it will (probably)
use your current working directory as its default location. If you use
your computer's start menu, it may want to save files in your desktop or
documents directory instead. You can change this by navigating to
another directory the first time you ``Save As\ldots{}''

\end{swcbox}

Let's type in a few lines of text, then use Control-O to write our
data to disk.  Once our file is saved, we can use Control-X to quit
the editor and return to the shell. (Unix documentation often uses the
shorthand \code{\^{}A} to mean ``control-A''.) \code{nano} doesn't
leave any output on the screen after it exits, but \code{ls} now
shows that we have created a file called \code{draft.txt}:

\begin{VerbIn}
$ ls
\end{VerbIn}

% [c]
\begin{VerbOut}
draft.txt
\end{VerbOut}

Let's tidy up by running \code{rm draft.txt}:

\begin{VerbIn}
$ rm draft.txt
\end{VerbIn}

\noindent
This command removes files (``rm'' is short for ``remove''). If we run
\code{ls} again, its output is empty once more, which tells us that
our file is gone:

\begin{VerbIn}
$ ls
\end{VerbIn}

\begin{swcbox}{Deleting Is Forever}

Unix doesn't have a trash bin: when we delete files, they are unhooked
from the file system so that their storage space on disk can be
recycled. Tools for finding and recovering deleted files do exist, but
there's no guarantee they'll work in any particular situation, since the
computer may recycle the file's disk space right away.

\end{swcbox}

Let's re-create that file and then move up one directory to
\code{/users/nelle} using \code{cd ..}:

\begin{VerbIn}
$ pwd
\end{VerbIn}

% [c]
\begin{VerbOut}
/users/nelle/thesis
\end{VerbOut}

% [c]
\begin{VerbIn}
$ nano draft.txt
$ ls
\end{VerbIn}

% [c]
\begin{VerbOut}
draft.txt
\end{VerbOut}

% [c]
\begin{VerbIn}
$ cd ..
\end{VerbIn}

If we try to remove the entire \code{thesis} directory using
\code{rm thesis}, we get an error message:

\begin{VerbIn}
$ rm thesis
\end{VerbIn}

% [c]
\begin{VerbErr}
rm: cannot remove `thesis': Is a directory
\end{VerbErr}

\noindent
This happens because \code{rm} only works on files, not directories.
The right command is \code{rmdir}, which is short for ``remove
directory''. It doesn't work yet either, though, because the directory
we're trying to remove isn't empty:

\begin{VerbIn}
$ rmdir thesis
\end{VerbIn}

% [c]
\begin{VerbErr}
rmdir: failed to remove `thesis': Directory not empty
\end{VerbErr}

This little safety feature can save you a lot of grief, particularly if
you are a bad typist. To really get rid of \code{thesis} we must first
delete the file \code{draft.txt}:

\begin{VerbIn}
$ rm thesis/draft.txt
\end{VerbIn}

The directory is now empty, so \code{rmdir} can delete it:

\begin{VerbIn}
$ rmdir thesis
\end{VerbIn}

\begin{swcbox}{With Great Power Comes Great Responsibility}

Removing the files in a directory just so that we can remove the
directory quickly becomes tedious. Instead, we can use \code{rm} with
the \code{-r} flag (which stands for ``recursive''):

\begin{VerbIn}
$ rm -r thesis
\end{VerbIn}

\noindent
This removes everything in the directory, then the directory itself. If
the directory contains sub-directories, \code{rm -r} does the same
thing to them, and so on. It's very handy, but can do a lot of damage if
used without care.

\end{swcbox}

Let's create that directory and file one more time. (Note that this time
we're running \code{nano} with the path \code{thesis/draft.txt},
rather than going into the \code{thesis} directory and running
\code{nano} on \code{draft.txt} there.)

\begin{VerbIn}
$ pwd
\end{VerbIn}

% [c]
\begin{VerbOut}
/users/nelle
\end{VerbOut}

% [c]
\begin{VerbIn}
$ mkdir thesis
\end{VerbIn}

% [c]
\begin{VerbIn}
$ nano thesis/draft.txt
$ ls thesis
\end{VerbIn}

% [c]
\begin{VerbOut}
draft.txt
\end{VerbOut}

\code{draft.txt} isn't a particularly informative name, so let's
change the file's name using \code{mv}, which is short for ``move'':

\begin{VerbIn}
$ mv thesis/draft.txt thesis/quotes.txt
\end{VerbIn}

\noindent
The first argument tells \code{mv} what we're ``moving'', while the
second is where it's to go. In this case, we're moving
\code{thesis/draft.txt} to \code{thesis/quotes.txt}, which has the
same effect as renaming the file. Sure enough, \code{ls} shows us that
\code{thesis} now contains one file called \code{quotes.txt}:

\begin{VerbIn}
$ ls thesis
\end{VerbIn}

% [c]
\begin{VerbOut}
quotes.txt
\end{VerbOut}

\noindent
(Just for the sake of inconsistency, \code{mv} also works on
directories---there is no separate \code{mvdir} command.)

Let's move \code{quotes.txt} into the current working directory. We
use \code{mv} once again, but this time we'll just use the name of a
directory as the second argument to tell \code{mv} that we want to
keep the filename, but put the file somewhere new. (This is why the
command is called ``move''.) In this case, the directory name we use is
the special directory name \code{.} that we mentioned earlier.

\begin{VerbIn}
$ mv thesis/quotes.txt .
\end{VerbIn}

\noindent
The effect is to move the file from the directory it was in to the
current working directory. \code{ls} now shows us that \code{thesis}
is empty:

\begin{VerbIn}
$ ls thesis
\end{VerbIn}

\noindent
Further, \code{ls} with a filename or directory name as an argument
only lists that file or directory. We can use this to see that
\code{quotes.txt} is still in our current directory:

\begin{VerbIn}
$ ls quotes.txt
\end{VerbIn}

% [c]
\begin{VerbOut}
quotes.txt
\end{VerbOut}

The \code{cp} command works very much like \code{mv}, except it
copies a file instead of moving it. We can check that it did the right
thing using \code{ls} with two paths as arguments---like most Unix
commands, \code{ls} can be given thousands of paths at once:

\begin{VerbIn}
$ cp quotes.txt thesis/quotations.txt
$ ls quotes.txt thesis/quotations.txt
\end{VerbIn}

% [c]
\begin{VerbOut}
quotes.txt   thesis/quotations.txt
\end{VerbOut}

To prove that we made a copy, let's delete the \code{quotes.txt} file
in the current directory and then run that same \code{ls} again. This
time it tells us that it can't find \code{quotes.txt} in the current
directory, but it does find the copy in \code{thesis} that we didn't
delete:

\begin{VerbIn}
$ ls quotes.txt thesis/quotations.txt
\end{VerbIn}

% [c]
\begin{VerbErr}
ls: cannot access quotes.txt: No such file or directory
\end{VerbErr}

% [c]
\begin{VerbOut}
thesis/quotations.txt
\end{VerbOut}

\begin{swcbox}{Another Useful Abbreviation}

The shell interprets the character \code{\textasciitilde{}} (tilde) at
the start of a path to mean ``the current user's home directory''. For
example, if Nelle's home directory is \code{/home/nelle}, then
\code{\textasciitilde{}/data} expands to
\code{/home/nelle/data}. This only works if it is the first character
in the path: \code{here/there/\textasciitilde{}/elsewhere} is
\emph{not} \code{/home/nelle/elsewhere}.

\end{swcbox}

\begin{keypoints}
\begin{swcitemize}
\item
  Unix documentation uses `\^{}A' to mean ``control-A''.
\item
  The shell does not have a trash bin: once something is deleted, it's
  really gone.
\item
  Nano is a very simple text editor---please use something else for real
  work.
\end{swcitemize}
\end{keypoints}

\begin{challenge}
  What is the output of the closing \code{ls} command in the sequence
  shown below?

\begin{VerbIn}
$ pwd
\end{VerbIn}

% [c]
\begin{VerbOut}
/home/thing/data
\end{VerbOut}

% [c]
\begin{VerbIn}
$ ls
\end{VerbIn}

% [c]
\begin{VerbOut}
proteins.dat
\end{VerbOut}

% [c]
\begin{VerbIn}
$ mkdir recombine
$ mv proteins.dat recombine
$ cp recombine/proteins.dat ../proteins-saved.dat
$ ls
\end{VerbIn}
\end{challenge}

\begin{challenge}
  Suppose that:

\begin{VerbIn}
$ ls -F
\end{VerbIn}

% [c]
\begin{VerbOut}
analyzed/  fructose.dat    raw/   sucrose.dat
\end{VerbOut}

  What command(s) could you run so that the commands below will produce
  the output shown?

\begin{VerbIn}
$ ls
\end{VerbIn}

% [c]
\begin{VerbOut}
analyzed   raw
\end{VerbOut}

% [c]
\begin{VerbIn}
$ ls analyzed
\end{VerbIn}

% [c]
\begin{VerbOut}
fructose.dat    sucrose.dat
\end{VerbOut}
\end{challenge}

\begin{challenge}
  What does \code{cp} do when given several filenames and a directory
  name, as in:

\begin{VerbIn}
$ mkdir backup
$ cp thesis/citations.txt thesis/quotations.txt backup
\end{VerbIn}

  What does it do when given three or more filenames, as in:

\begin{VerbIn}
$ ls -F
\end{VerbIn}

% [c]
\begin{VerbOut}
intro.txt    methods.txt    survey.txt
\end{VerbOut}

% [c]
\begin{VerbIn}
$ cp intro.txt methods.txt survey.txt
\end{VerbIn}
\end{challenge}

\begin{challenge}
  The command \code{ls -R} lists the contents of directories
  recursively, i.e., lists their sub-directories, sub-sub-directories,
  and so on in alphabetical order at each level. The command
  \code{ls -t} lists things by time of last change, with most recently
  changed files or directories first. In what order does
  \code{ls -R -t} display things?
\end{challenge}

\section{Pipes and Filters}

\begin{objectives}
\begin{swcitemize}
\item
  Redirect a command's output to a file.
\item
  Process a file instead of keyboard input using redirection.
\item
  Construct command pipelines with two or more stages.
\item
  Explain what usually happens if a program or pipeline isn't given any
  input to process.
\item
  Explain Unix's ``small pieces, loosely joined'' philosophy.
\end{swcitemize}
\end{objectives}

Now that we know a few basic commands, we can finally look at the
shell's most powerful feature: the ease with which it lets us combine
existing programs in new ways. We'll start with a directory called
\code{molecules} that contains six files describing some simple
organic molecules. The \code{.pdb} extension indicates that these
files are in Protein Data Bank format, a simple text format that
specifies the type and position of each atom in the molecule.

\begin{VerbIn}
$ ls molecules
\end{VerbIn}

% [c]
\begin{VerbOut}
cubane.pdb    ethane.pdb    methane.pdb
octane.pdb    pentane.pdb   propane.pdb
\end{VerbOut}

Let's go into that directory with \code{cd} and run the command
\code{wc *.pdb}. \code{wc} is the ``word count'' command: it counts
the number of lines, words, and characters in files. The \code{*} in
\code{*.pdb} matches zero or more characters, so the shell turns
\code{*.pdb} into a complete list of \code{.pdb} files:

% [c]
\begin{VerbIn}
$ cd molecules
$ wc *.pdb
\end{VerbIn}

% [c]
\begin{VerbOut}
  20  156 1158 cubane.pdb
  12   84  622 ethane.pdb
   9   57  422 methane.pdb
  30  246 1828 octane.pdb
  21  165 1226 pentane.pdb
  15  111  825 propane.pdb
 107  819 6081 total
\end{VerbOut}

\begin{swcbox}{Wildcards}

\code{*} is a \gl{wildcard}{g:wildcard}. It matches zero or more
characters, so \code{*.pdb} matches \code{ethane.pdb},
\code{propane.pdb}, and so on. On the other hand, \code{p*.pdb} only
matches \code{pentane.pdb} and \code{propane.pdb}, because the `p'
at the front only matches itself.

\code{?} is also a wildcard, but it only matches a single character.
This means that \code{p?.pdb} matches \code{pi.pdb} or
\code{p5.pdb}, but not \code{propane.pdb}. We can use any number of
wildcards at a time: for example, \code{p*.p?*} matches anything that
starts with a `p' and ends with `.', `p', and at least one more
character (since the `?' has to match one character, and the final `*'
can match any number of characters). Thus, \code{p*.p?*} would match
\code{preferred.practice}, and even \code{p.pi} (since the first `*'
can match no characters at all), but not \code{quality.practice}
(doesn't start with `p') or \code{preferred.p} (there isn't at least
one character after the `.p').

When the shell sees a wildcard, it expands the wildcard to create a list
of matching filenames \emph{before} running the command that was asked
for. This means that commands like \code{wc} and \code{ls} never see
the wildcard characters, just what those wildcards matched. This is
another example of orthogonal design.

\end{swcbox}

If we run \code{wc -l} instead of just \code{wc}, the output shows
only the number of lines per file:

\begin{VerbIn}
$ wc -l *.pdb
\end{VerbIn}

% [c]
\begin{VerbOut}
  20  cubane.pdb
  12  ethane.pdb
   9  methane.pdb
  30  octane.pdb
  21  pentane.pdb
  15  propane.pdb
 107  total
\end{VerbOut}

\noindent
We can also use \code{-w} to get only the number of words, or
\code{-c} to get only the number of characters.

Which of these files is shortest? It's an easy question to answer when
there are only six files, but what if there were 6000? Our first step
toward a solution is to run the command:

\begin{VerbIn}
$ wc -l *.pdb > lengths
\end{VerbIn}

The \code{\textgreater{}} tells the shell to
\gl{redirect}{g:redirect} the command's output to a file instead
of printing it to the screen. The shell will create the file if it
doesn't exist, or overwrite the contents of that file if it does. (This
is why there is no screen output: everything that \code{wc} would have
printed has gone into the file \code{lengths} instead.)
\code{ls lengths} confirms that the file exists:

\begin{VerbIn}
$ ls lengths
\end{VerbIn}

% [c]
\begin{VerbOut}
lengths
\end{VerbOut}

We can now send the content of \code{lengths} to the screen using
\code{cat lengths}. \code{cat} stands for ``concatenate'': it prints
the contents of files one after another. There's only one file in this
case, so \code{cat} just shows us what it contains:

\begin{VerbIn}
$ cat lengths
\end{VerbIn}

% [c]
\begin{VerbOut}
  20  cubane.pdb
  12  ethane.pdb
   9  methane.pdb
  30  octane.pdb
  21  pentane.pdb
  15  propane.pdb
 107  total
\end{VerbOut}

Now let's use the \code{sort} command to sort its contents.  We will
also use the \code{-n} flag to specify that the sort is numerical
instead of alphabetical.  This does \emph{not} change the file;
instead, it sends the sorted result to the screen:

\begin{VerbIn}
$ sort -n lengths
\end{VerbIn}

% [c]
\begin{VerbOut}
  9  methane.pdb
 12  ethane.pdb
 15  propane.pdb
 20  cubane.pdb
 21  pentane.pdb
 30  octane.pdb
107  total
\end{VerbOut}

We can put the sorted list of lines in another temporary file called
\code{sorted-lengths} by putting
\code{\textgreater{} sorted-lengths} after the command, just as we
used \code{\textgreater{} lengths} to put the output of \code{wc}
into \code{lengths}. Once we've done that, we can run another command
called \code{head} to get the first few lines in
\code{sorted-lengths}:

\begin{VerbIn}
$ sort -n lengths > sorted-lengths
$ head -1 sorted-lengths
\end{VerbIn}

% [c]
\begin{VerbOut}
  9  methane.pdb
\end{VerbOut}

Using the argument \code{-1} with \code{head} tells it that we only
want the first line of the file; \code{-20} would get the first 20,
and so on. Since \code{sorted-lengths} contains the lengths of our
files ordered from least to greatest, the output of \code{head} must
be the file with the fewest lines.

If you think this is confusing, you're in good company: even once you
understand what \code{wc}, \code{sort}, and \code{head} do, all
those intermediate files make it hard to follow what's going on. We can
make it easier to understand by running \code{sort} and \code{head}
together:

\begin{VerbIn}
$ sort -n lengths | head -1
\end{VerbIn}

% [c]
\begin{VerbOut}
  9  methane.pdb
\end{VerbOut}

The vertical bar between the two commands is called a
\gl{pipe}{g:pipe}. It tells the shell that we want to use the
output of the command on the left as the input to the command on the
right. The computer might create a temporary file if it needs to, or
copy data from one program to the other in memory, or something else
entirely; we don't have to know or care.

We can use another pipe to send the output of \code{wc} directly to
\code{sort}, which then sends its output to \code{head}:

\begin{VerbIn}
$ wc -l *.pdb | sort -n | head -1
\end{VerbIn}

% [c]
\begin{VerbOut}
  9  methane.pdb
\end{VerbOut}

\noindent
This is exactly like a mathematician nesting functions like
\emph{sin($\pi$x)\textsuperscript{2}} and saying ``the square of the sine of
\emph{x} times $\pi$''. In our case, the calculation is ``head of sort of
line count of \code{*.pdb}''.

Here's what actually happens behind the scenes when we create a pipe.
When a computer runs a program---any program---it creates a
\gl{process}{g:process} in memory to hold the program's software
and its current state. Every process has an input channel called
\gl{standard input}{g:standard-input}. (By this point, you may be
surprised that the name is so memorable, but don't worry: most Unix
programmers call it ``stdin''.) Every process also has a default output
channel called \gl{standard output}{g:standard-output} (or
``stdout'').

The shell is actually just another program. Under normal circumstances,
whatever we type on the keyboard is sent to the shell on its standard
input, and whatever it produces on standard output is displayed on our
screen. When we tell the shell to run a program, it creates a new
process and temporarily sends whatever we type on our keyboard to that
process's standard input, and whatever the process sends to standard
output to the screen.

When we run
\code{wc -l *.pdb \textgreater{} lengths}, the shell starts by telling
the computer to create a new process to run the \code{wc} program.
Since we've provided some filenames as arguments, \code{wc} reads
from them instead of from standard input. And since we've used
\code{\textgreater{}} to redirect output to a file, the shell connects
the process's standard output to that file.

If we run \code{wc -l *.pdb \textbar{} sort -n} instead, the shell
creates two processes (one for each process in the pipe) so that
\code{wc} and \code{sort} run simultaneously. The standard output of
\code{wc} is fed directly to the standard input of \code{sort};
since there's no redirection with \code{\textgreater{}},
\code{sort}'s output goes to the screen. And if we run
\code{wc -l *.pdb \textbar{} sort -n \textbar{} head -1}, we get three
processes with data flowing from the files, through \code{wc} to
\code{sort}, and from \code{sort} through \code{head} to the
screen.

This simple idea is why Unix has been so successful. Instead of creating
enormous programs that try to do many different things, Unix programmers
focus on creating lots of simple tools that each do one job well, and
that work well with each other. This programming model is called
\gl{pipes and filters}{g:pipe-and-filter}. We've already seen
pipes; a \gl{filter}{g:filter} is a program like \code{wc} or
\code{sort} that transforms a stream of input into a stream of output.
Almost all of the standard Unix tools can work this way: unless told to
do otherwise, they read from standard input, do something with what
they've read, and write to standard output.

The key is that any program that reads lines of text from standard input
and writes lines of text to standard output can be combined with every
other program that behaves this way as well. You can \emph{and should}
write your programs this way so that you and other people can put those
programs into pipes to multiply their power.

\begin{swcbox}{Redirecting Input}

% FIXME: delay this until it's used.

As well as using \code{\textgreater{}} to redirect a program's output,
we can use \code{\textless{}} to redirect its input, i.e., to read
from a file instead of from standard input. For example, instead of
writing \code{wc ammonia.pdb}, we could write
\code{wc \textless{} ammonia.pdb}. In the first case, \code{wc} gets
a command line argument telling it what file to open. In the second,
\code{wc} doesn't have any command line arguments, so it reads from
standard input, but we have told the shell to send the contents of
\code{ammonia.pdb} to \code{wc}'s standard input.

\end{swcbox}

\subsection*{Nelle's Pipeline: Checking Files}

Nelle has run her samples through the assay machines and created 1520
files in the \code{north-pacific-gyre/2012-07-03} directory described
earlier. As a quick sanity check, she types:

\begin{VerbIn}
$ cd north-pacific-gyre/2012-07-03
$ wc -l *.txt
\end{VerbIn}

\noindent
The output is 1520 lines that look like this:

\begin{VerbOut}
300 NENE01729A.txt
300 NENE01729B.txt
300 NENE01736A.txt
300 NENE01751A.txt
300 NENE01751B.txt
300 NENE01812A.txt
...
\end{VerbOut}

Now she types this:

\begin{VerbIn}
$ wc -l *.txt | sort -n | head -5
\end{VerbIn}

% [c]
\begin{VerbOut}
 240 NENE02018B.txt
 300 NENE01729A.txt
 300 NENE01729B.txt
 300 NENE01736A.txt
 300 NENE01751A.txt
\end{VerbOut}

\noindent
Whoops: one of the files is 60 lines shorter than the others. When she
goes back and checks it, she sees that she did that assay at 8:00 on a
Monday morning---someone was probably in using the machine on the
weekend, and she forgot to reset it. Before re-running that sample, she
checks to see if any files have too much data:

\begin{VerbIn}
$ wc -l *.txt | sort -n | tail -5
\end{VerbIn}

% [c]
\begin{VerbOut}
 300 NENE02040A.txt
 300 NENE02040B.txt
 300 NENE02040Z.txt
 300 NENE02043A.txt
 300 NENE02043B.txt
\end{VerbOut}

Those numbers look good---but what's that `Z' doing there in the
third-to-last line? All of her samples should be marked `A' or `B'; by
convention, her lab uses `Z' to indicate samples with missing
information. To find others like it, she does this:

\begin{VerbIn}
$ ls *Z.txt
\end{VerbIn}

% [c]
\begin{VerbOut}
NENE01971Z.txt    NENE02040Z.txt
\end{VerbOut}

Sure enough, when she checks the log on her laptop, there's no depth
recorded for either of those samples. Since it's too late to get the
information any other way, she must exclude those two files from her
analysis. She could just delete them using \code{rm}, but there are
actually some analyses she might do later where depth doesn't matter, so
instead, she'll just be careful later on to select files using the
wildcard expression \code{*{[}AB{]}.txt}. As always, the `*' matches
any number of characters; the expression \code{{[}AB{]}} matches
either an `A' or a `B', so this matches all the valid data files she
has.

\begin{keypoints}
\begin{swcitemize}
\item
  \code{command \textgreater{} file} redirects a command's output to a
  file.
\item
  \code{first \textbar{} second} is a pipeline: the output of the
  first command is used as the input to the second.
\item
  The best way to use the shell is to use pipes to combine simple
  single-purpose programs (filters).
\end{swcitemize}
\end{keypoints}

\begin{challenge}
  If we run \code{sort} on this file:

\begin{VerbOut}
10
2
19
22
6
\end{VerbOut}

  \noindent
  the output is:

\begin{VerbOut}
10
19
2
22
6
\end{VerbOut}

  \noindent
  If we run \code{sort -n} on the same input, we get this instead:

\begin{VerbOut}
2
6
10
19
22
\end{VerbOut}

  \noindent
  Explain why \code{-n} has this effect.
\end{challenge}

\begin{challenge}
  What is the difference between:

\begin{VerbIn}
$ wc -l < mydata.dat
\end{VerbIn}

  \noindent
  and:

\begin{VerbIn}
$ wc -l mydata.dat
\end{VerbIn}
\end{challenge}

\begin{challenge}
  The command \code{uniq} removes adjacent duplicated lines from its
  input. For example, if a file \code{salmon.txt} contains:

\begin{VerbOut}
coho
coho
steelhead
coho
steelhead
steelhead
\end{VerbOut}

  \noindent
  then \code{uniq salmon.txt} produces:

\begin{VerbOut}
coho
steelhead
coho
steelhead
\end{VerbOut}

  \noindent
  Why do you think \code{uniq} only removes \emph{adjacent} duplicated
  lines? (Hint: think about very large data sets.) What other command
  could you combine with it in a pipe to remove all duplicated lines?
\end{challenge}

\begin{challenge}
  A file called \code{animals.txt} contains the following data:

\begin{VerbOut}
2012-11-05,deer
2012-11-05,rabbit
2012-11-05,raccoon
2012-11-06,rabbit
2012-11-06,deer
2012-11-06,fox
2012-11-07,rabbit
2012-11-07,bear
\end{VerbOut}

  What text passes through each of the pipes and the final redirect in
  the pipeline below?

\begin{VerbIn}
$ cat animals.txt | head -5 | tail -3 | sort -r > final.txt
\end{VerbIn}
\end{challenge}

\begin{challenge}
  The command:

\begin{VerbIn}
$ cut -d , -f 2 animals.txt
\end{VerbIn}

  \noindent
  produces:

\begin{VerbOut}
deer
rabbit
raccoon
rabbit
deer
fox
rabbit
bear
\end{VerbOut}

  \noindent
  What other command(s) could be added to this in a pipeline to find out
  what animals the file contains (without any duplicates in their
  names)?
\end{challenge}

\section{Loops}

\begin{objectives}
\begin{swcitemize}
\item
  Write a loop that applies one or more commands separately to each file
  in a set of files.
\item
  Trace the values taken on by a loop variable during execution of the
  loop.
\item
  Explain the difference between a variable's name and its value.
\item
  Explain why spaces and some punctuation characters shouldn't be used
  in files' names.
\item
  Demonstrate how to see what commands have recently been executed.
\item
  Re-run recently executed commands without retyping them.
\end{swcitemize}
\end{objectives}

Wildcards and tab completion are two ways to reduce typing (and typing
mistakes). Another is to tell the shell to do something over and over
again. Suppose we have several hundred genome data files named
\code{basilisk.dat}, \code{unicorn.dat}, and so on. When new files
arrive, we'd like to rename the existing ones to
\code{original-basilisk.dat} and \code{original-unicorn.dat}. We
can't use:

\begin{VerbIn}
$ mv *.dat original-*.dat
\end{VerbIn}

\noindent
because that would expand (in the two-file case) to:

\begin{VerbIn}
$ mv basilisk.dat unicorn.dat
\end{VerbIn}

This wouldn't back up our files: it would replace the content of
\code{unicorn.dat} with whatever's in \code{basilisk.dat}.
Instead, we can use a \gl{loop}{g:for-loop} to do some operation
once for each thing in a list. Here's a simple example that displays the
first three lines of each file in turn:

\begin{VerbIn}
$ for filename in basilisk.dat unicorn.dat
> do
>    head -3 $filename
> done
\end{VerbIn}

% [c]
\begin{VerbOut}
COMMON NAME: basilisk
CLASSIFICATION: basiliscus vulgaris
UPDATED: 1745-05-02
COMMON NAME: unicorn
CLASSIFICATION: equus monoceros
UPDATED: 1738-11-24
\end{VerbOut}

When the shell sees the keyword \code{for}, it knows it is supposed to
repeat a command (or group of commands) once for each thing in a list.
In this case, the list is the two filenames. Each time through the loop,
the name of the thing currently being operated on is assigned to the
\gl{variable}{g:variable} called \code{filename}. Inside the
loop, we get the variable's value by putting \code{\$} in front of it:
\code{\$filename} is \code{basilisk.dat} the first time through the
loop, \code{unicorn.dat} the second, and so on. Finally, the command
that's actually being run is our old friend \code{head}, so this loop
prints out the first three lines of each data file in turn.

\begin{swcbox}{Follow the Prompt}

The shell prompt changes from \code{\$} to \code{\textgreater{}} and
back again as we were typing in our loop. The second prompt,
\code{\textgreater{}}, is different to remind us that we haven't
finished typing a complete command yet.

\end{swcbox}

We have called the variable in this loop \code{filename} in order to
make its purpose clearer to human readers. The shell itself doesn't care
what the variable is called; if we wrote this loop as:

\begin{VerbIn}
for x in basilisk.dat unicorn.dat
do
    head -3 $x
done
\end{VerbIn}

\noindent
or:

\begin{VerbIn}
for temperature in basilisk.dat unicorn.dat
do
    head -3 $temperature
done
\end{VerbIn}

\noindent
it would work exactly the same way. \emph{Don't do this.} Programs are
only useful if people can understand them, so meaningless names (like
\code{x}) or misleading names (like \code{temperature}) increase the
odds that the program won't do what its readers think it does.

Here's a slightly more complicated loop:

\begin{VerbIn}
for filename in *.dat
do
    echo $filename
    head -100 $filename | tail -20
done
\end{VerbIn}

The shell starts by expanding \code{*.dat} to create the list of files
it will process. The \gl{loop body}{g:loop-body} then executes two
commands for each of those files. The first, \code{echo}, just prints
its command-line arguments to standard output. For example:

\begin{VerbIn}
$ echo hello there
\end{VerbIn}

\noindent
prints:

\begin{VerbOut}
hello there
\end{VerbOut}

In this case, since the shell expands \code{\$filename} to be the name
of a file, \code{echo \$filename} just prints the name of the file.
Note that we can't write this as:

\begin{VerbIn}
for filename in *.dat
do
    $filename
    head -100 $filename | tail -20
done
\end{VerbIn}

\noindent
because then the first time through the loop, when \code{\$filename}
expanded to \code{basilisk.dat}, the shell would try to run
\code{basilisk.dat} as a program. Finally, the \code{head} and
\code{tail} combination selects lines 81-100 from whatever file is
being processed.

\begin{swcbox}{Spaces in Names}

Filename expansion in loops is another reason you should not use spaces
in filenames. Suppose our data files are named:

\begin{VerbFile}
basilisk.dat
red dragon.dat
unicorn.dat
\end{VerbFile}

\noindent
If we try to process them using:

\begin{VerbIn}
for filename in *.dat
do
    head -100 $filename | tail -20
done
\end{VerbIn}

\noindent
then the shell will expand \code{*.dat} to create:

\begin{VerbOut}
basilisk.dat red dragon.dat unicorn.dat
\end{VerbOut}

With older versions of Bash, or most other shells, \code{filename}
will then be assigned the following values in turn:

\begin{VerbFile}
basilisk.dat
red
dragon.dat
unicorn.dat
\end{VerbFile}

That's a problem: \code{head} can't read files called \code{red} and
\code{dragon.dat} because they don't exist, and won't be asked to read
the file \code{red dragon.dat}.

We can make our script a little bit more robust by
\gl{quoting}{g:shell-quoting} our use of the variable:

\begin{VerbIn}
for filename in *.dat
do
    head -100 "$filename" | tail -20
done
\end{VerbIn}

\noindent
but it's simpler just to avoid using spaces (or other special
characters) in filenames.

\end{swcbox}

Going back to our original file renaming problem, we can solve it using
this loop:

\begin{VerbIn}
for filename in *.dat
do
    mv $filename original-$filename
done
\end{VerbIn}

\noindent
This loop runs the \code{mv} command once for each filename. The first
time, when \code{\$filename} expands to \code{basilisk.dat}, the
shell executes:

\begin{VerbIn}
mv basilisk.dat original-basilisk.dat
\end{VerbIn}

\noindent
The second time, the command is:

\begin{VerbIn}
mv unicorn.dat original-unicorn.dat
\end{VerbIn}

\begin{swcbox}{Measure Twice, Run Once}

A loop is a way to do many things at once---or to make many mistakes at
once if it does the wrong thing. One way to check what a loop
\emph{would} do is to echo the commands it would run instead of actually
running them. For example, we could write our file renaming loop like
this:

\begin{VerbIn}
for filename in *.dat
do
    echo mv $filename original-$filename
done
\end{VerbIn}

Instead of running \code{mv}, this loop runs \code{echo}, which
prints out:

\begin{VerbIn}
mv basilisk.dat original-basilisk.dat
mv unicorn.dat original-unicorn.dat
\end{VerbIn}

\noindent
\emph{without} actually running those commands. We can then use up-arrow
to redisplay the loop, back-arrow to get to the word \code{echo},
delete it, and then press ``enter'' to run the loop with the actual
\code{mv} commands. This isn't foolproof, but it's a handy way to see
what's going to happen when you're still learning how loops work.

\end{swcbox}

\subsection*{Nelle's Pipeline: Processing Files}

Nelle is now ready to process her data files. Since she's still learning
how to use the shell, she decides to build up the required commands in
stages. Her first step is to make sure that she can select the right
files---remember, these are ones whose names end in `A' or `B', rather
than `Z':

\begin{VerbIn}
$ cd north-pacific-gyre/2012-07-03
$ for datafile in *[AB].txt
> do
>     echo $datafile
> done
\end{VerbIn}

% [c]
\begin{VerbOut}
NENE01729A.txt
NENE01729B.txt
NENE01736A.txt
...
NENE02043A.txt
NENE02043B.txt
\end{VerbOut}

Her next step is to decide what to call the files that the
\code{goostats} analysis program will create. Prefixing each input
file's name with ``stats'' seems simple, so she modifies her loop to do
that:

\begin{VerbIn}
$ for datafile in *[AB].txt
> do
>     echo $datafile stats-$datafile
> done
\end{VerbIn}

% [c]
\begin{VerbOut}
NENE01729A.txt stats-NENE01729A.txt
NENE01729B.txt stats-NENE01729B.txt
NENE01736A.txt stats-NENE01736A.txt
...
NENE02043A.txt stats-NENE02043A.txt
NENE02043B.txt stats-NENE02043B.txt
\end{VerbOut}

She hasn't actually run \code{goostats} yet, but now she's sure she
can select the right files and generate the right output filenames.

Typing in commands over and over again is becoming tedious, though, and
Nelle is worried about making mistakes, so instead of re-entering her
loop, she presses the up arrow. In response, the shell redisplays the
whole loop on one line (using semi-colons to separate the pieces):

\begin{VerbIn}
$ for datafile in *[AB].txt; do echo $datafile stats-$datafile; done
\end{VerbIn}

Using the left arrow key, Nelle backs up and changes the command
\code{echo} to \code{goostats}:

\begin{VerbIn}
$ for datafile in *[AB].txt; do bash goostats $datafile stats-$datafile; done
\end{VerbIn}

When she presses enter, the shell runs the modified command. However,
nothing appears to happen---there is no output. After a moment, Nelle
realizes that since her script doesn't print anything to the screen any
longer, she has no idea whether it is running, much less how quickly.
She kills the job by typing Control-C, uses up-arrow to repeat the
command, and edits it to read:

\begin{VerbIn}
$ for datafile in *[AB].txt; do echo $datafile; bash goostats $datafile stats-$datafile; done
\end{VerbIn}

\begin{swcbox}{Beginning and End}

We can move to the beginning of a line in the shell by typing
\code{\^{}A} (which means Control-A) and to the end using
\code{\^{}E}.

\end{swcbox}

When she runs her program now, it produces one line of output every five
seconds or so:

\begin{VerbOut}
NENE01729A.txt
NENE01729B.txt
NENE01736A.txt
...
\end{VerbOut}

1518 times 5 seconds, divided by 60, tells her that her script will take
about two hours to run. As a final check, she opens another terminal
window, goes into \code{north-pacific-gyre/2012-07-03}, and uses
\code{cat stats-NENE01729B.txt} to examine one of the output files. It
looks good, so she decides to get some coffee and catch up on her
reading.

\begin{swcbox}{Those Who Know History Can Choose to Repeat It}

Another way to repeat previous work is to use the \code{history}
command to get a list of the last few hundred commands that have been
executed, and then to use \code{!123} (where ``123'' is replaced by
the command number) to repeat one of those commands. For example, if
Nelle types this:

\begin{VerbIn}
$ history | tail -5
\end{VerbIn}

% [c]
\begin{VerbOut}
  456  ls -l NENE0*.txt
  457  rm stats-NENE01729B.txt.txt
  458  bash goostats NENE01729B.txt stats-NENE01729B.txt
  459  ls -l NENE0*.txt
  460  history
\end{VerbOut}

\noindent
then she can re-run \code{goostats} on \code{NENE01729B.txt} simply
by typing \code{!458}.

\end{swcbox}

\begin{keypoints}
\begin{swcitemize}
\item
  A \code{for} loop repeats commands once for every thing in a list.
\item
  Every \code{for} loop needs a variable to refer to the current
  ``thing''.
\item
  Use \code{\$name} to expand a variable (i.e., get its value).
\item
  Do not use spaces, quotes, or wildcard characters such as `*' or `?'
  in filenames, as it complicates variable expansion.
\item
  Give files consistent names that are easy to match with wildcard
  patterns to make it easy to select them for looping.
\item
  Use the up-arrow key to scroll up through previous commands to edit
  and repeat them.
\item
  Use \code{history} to display recent commands, and \code{!number}
  to repeat a command by number.
\end{swcitemize}
\end{keypoints}

\begin{challenge}
  Suppose that \code{ls} initially displays:

\begin{VerbOut}
fructose.dat    glucose.dat   sucrose.dat
\end{VerbOut}

  \noindent
  What is the output of:

\begin{VerbIn}
for datafile in *.dat
do
    ls *.dat
done
\end{VerbIn}
\end{challenge}

\begin{challenge}
  In the same directory, what is the effect of this loop?

\begin{VerbIn}
for sugar in *.dat
do
    echo $sugar
    cat $sugar > xylose.dat
done
\end{VerbIn}

  \begin{swcenumerate}
  \item
    Prints \code{fructose.dat}, \code{glucose.dat}, and
    \code{sucrose.dat}, and copies \code{sucrose.dat} to create
    \code{xylose.dat}.
  \item
    Prints \code{fructose.dat}, \code{glucose.dat}, and
    \code{sucrose.dat}, and concatenates all three files to create
    \code{xylose.dat}.
  \item
    Prints \code{fructose.dat}, \code{glucose.dat},
    \code{sucrose.dat}, and \code{xylose.dat}, and copies
    \code{sucrose.dat} to create \code{xylose.dat}.
  \item
    None of the above.
  \end{swcenumerate}
\end{challenge}

\begin{challenge}
  The \code{expr} program does simple arithmetic using command-line
  arguments:

\begin{VerbIn}
$ expr 3 + 5
\end{VerbIn}

% [c]
\begin{VerbOut}
8
\end{VerbOut}

% [c]
\begin{VerbIn}
$ expr 30 / 5 - 2
\end{VerbIn}

% [c]
\begin{VerbOut}
4
\end{VerbOut}

  \noindent
  Given this, what is the output of:

\begin{VerbIn}
for left in 2 3
do
    for right in $left
    do
        expr $left + $right
    done
done
\end{VerbIn}
\end{challenge}

\begin{challenge}
  Describe in words what the following loop does.

\begin{VerbIn}
for how in frog11 prcb redig
do
    $how -limit 0.01 NENE01729B.txt
done
\end{VerbIn}
\end{challenge}

\section{Shell Scripts}

\begin{objectives}
\begin{swcitemize}
\item
  Write a shell script that runs a command or series of commands for a
  fixed set of files.
\item
  Run a shell script from the command line.
\item
  Write a shell script that operates on a set of files defined by the
  user on the command line.
\item
  Create pipelines that include user-written shell scripts.
\end{swcitemize}
\end{objectives}

We are finally ready to see what makes the shell such a powerful
programming environment. We are going to take the commands we repeat
frequently and save them in files so that we can re-run all those
operations again later by typing a single command. For historical
reasons, a bunch of commands saved in a file is usually called a
\gl{shell script}{g:shell-script}, but make no mistake: these are
actually small programs.

Let's start by putting the following line in the file
\code{middle.sh}:

\begin{VerbFile}
head -20 cholesterol.pdb | tail -5
\end{VerbFile}

This is a variation on the pipe we constructed earlier: it selects lines
16-20 of the file \code{cholesterol.pdb}. Remember, we are \emph{not}
running it as a command just yet: we are putting the commands in a file.

Once we have saved the file, we can ask the shell to execute the
commands it contains. Our shell is called \code{bash}, so we run the
following command:

\begin{VerbIn}
$ bash middle.sh
\end{VerbIn}

% [c]
\begin{VerbOut}
ATOM     14  C           1      -1.463  -0.666   1.001  1.00  0.00
ATOM     15  C           1       0.762  -0.929   0.295  1.00  0.00
ATOM     16  C           1       0.771  -0.937   1.840  1.00  0.00
ATOM     17  C           1      -0.664  -0.610   2.293  1.00  0.00
ATOM     18  C           1      -4.705   2.108  -0.396  1.00  0.00
\end{VerbOut}

Sure enough, our script's output is exactly what we would get if we ran
that pipeline directly.

\begin{swcbox}{Text vs.\ Whatever}

We usually call programs like Microsoft Word or LibreOffice Writer
``text editors'', but we need to be a bit more careful when it comes to
programming. By default, Microsoft Word uses \code{.docx} files to
store not only text, but also formatting information about fonts,
headings, and so on. This extra information isn't stored as characters,
and doesn't mean anything to tools like \code{head}: they expect input
files to contain nothing but the letters, digits, and punctuation on a
standard computer keyboard. When editing programs, therefore, you must
either use a plain text editor, or be careful to save files as plain
text.

\end{swcbox}

What if we want to select lines from an arbitrary file? We could edit
\code{middle.sh} each time to change the filename, but that would
probably take longer than just retyping the command. Instead, let's edit
\code{middle.sh} and replace \code{cholesterol.pdb} with a special
variable called \code{\$1}:

\begin{VerbIn}
$ cat middle.sh
\end{VerbIn}

% [c]
\begin{VerbFile}
head -20 $1 | tail -5
\end{VerbFile}

Inside a shell script, \code{\$1} means ``the first filename (or other
argument) on the command line''. We can now run our script like this:

\begin{VerbIn}
$ bash middle.sh cholesterol.pdb
\end{VerbIn}

% [c]
\begin{VerbOut}
ATOM     14  C           1      -1.463  -0.666   1.001  1.00  0.00
ATOM     15  C           1       0.762  -0.929   0.295  1.00  0.00
ATOM     16  C           1       0.771  -0.937   1.840  1.00  0.00
ATOM     17  C           1      -0.664  -0.610   2.293  1.00  0.00
ATOM     18  C           1      -4.705   2.108  -0.396  1.00  0.00
\end{VerbOut}

\noindent
or on a different file like this:

\begin{VerbIn}
$ bash middle.sh vitamin-a.pdb
\end{VerbIn}

\begin{VerbOut}
ATOM     14  C           1       1.788  -0.987  -0.861
ATOM     15  C           1       2.994  -0.265  -0.829
ATOM     16  C           1       4.237  -0.901  -1.024
ATOM     17  C           1       5.406  -0.117  -1.087
ATOM     18  C           1      -0.696  -2.628  -0.641
\end{VerbOut}

We still need to edit \code{middle.sh} each time we want to adjust the
range of lines, though. Let's fix that by using the special variables
\code{\$2} and \code{\$3}:

\begin{VerbIn}
$ cat middle.sh
\end{VerbIn}

% [c]
\begin{VerbFile}
head $2 $1 | tail $3
\end{VerbFile}

% [c]
\begin{VerbIn}
$ bash middle.sh vitamin-a.pdb -20 -5
\end{VerbIn}

% [c]
\begin{VerbOut}
ATOM     14  C           1       1.788  -0.987  -0.861
ATOM     15  C           1       2.994  -0.265  -0.829
ATOM     16  C           1       4.237  -0.901  -1.024
ATOM     17  C           1       5.406  -0.117  -1.087
ATOM     18  C           1      -0.696  -2.628  -0.641
\end{VerbOut}

This works, but it may take the next person who reads \code{middle.sh}
a moment to figure out what it does. We can improve our script by adding
some \gl{comments}{g:comment} at the top:

\begin{VerbIn}
$ cat middle.sh
\end{VerbIn}

% [c]
\begin{VerbOut}
# Select lines from the middle of a file.
# Usage: middle.sh filename -end_line -num_lines
head $2 $1 | tail $3
\end{VerbOut}

A comment starts with a \code{\#} character and runs to the end of the
line. The computer ignores comments, but they're invaluable for helping
people understand and use scripts.

What if we want to process many files in a single pipeline? For example,
if we want to sort our \code{.pdb} files by length, we would type:

\begin{VerbIn}
$ wc -l *.pdb | sort -n
\end{VerbIn}

\noindent
because \code{wc -l} lists the number of lines in the files and
\code{sort -n} sorts things numerically. We could put this in a file,
but then it would only ever sort a list of \code{.pdb} files in the
current directory. If we want to be able to get a sorted list of other
kinds of files, we need a way to get all those names into the script. We
can't use \code{\$1}, \code{\$2}, and so on because we don't know
how many files there are. Instead, we use the special variable
\code{\$*}, which means, ``All of the command-line arguments to the
shell script.'' Here's an example:

\begin{VerbIn}
$ cat sorted.sh
\end{VerbIn}

% [c]
\begin{VerbOut}
wc -l $* | sort -n
\end{VerbOut}

% [c]
\begin{VerbIn}
$ bash sorted.sh *.dat backup/*.dat
\end{VerbIn}

% [c]
\begin{VerbOut}
      29 chloratin.dat
      89 backup/chloratin.dat
      91 sphagnoi.dat
     156 sphag2.dat
     172 backup/sphag-merged.dat
     182 girmanis.dat
\end{VerbOut}

\begin{swcbox}{Why Isn't It Doing Anything?}

What happens if a script is supposed to process a bunch of files, but we
don't give it any filenames? For example, what if we type:

\begin{VerbIn}
$ bash sorted.sh
\end{VerbIn}

\noindent
but don't add \code{*.dat} (or anything else)? In this case,
\code{\$*} expands to nothing at all, so the pipeline inside the
script is effectively:

\begin{VerbFile}
wc -l | sort -n
\end{VerbFile}

Since it doesn't have any filenames, \code{wc} assumes it is supposed
to process standard input, so it just sits there and waits for us to
give it some data interactively. From the outside, though, all we see is
it sitting there: the script doesn't appear to do anything.

\end{swcbox}

We have two more things to do before we're finished with our simple
shell scripts. If you look at a script like:

\begin{VerbFile}
wc -l $* | sort -n
\end{VerbFile}

\noindent
you can probably puzzle out what it does. On the other hand, if you look
at this script:

\begin{VerbFile}
# List files sorted by number of lines.
wc -l $* | sort -n
\end{VerbFile}

\noindent
you don't have to puzzle it out---the comment at the top tells you what
it does. A line or two of documentation like this make it much easier
for other people (including your future self) to re-use your work. The
only caveat is that each time you modify the script, you should check
that the comment is still accurate: an explanation that sends the reader
in the wrong direction is worse than none at all.

Second, suppose we have just run a series of commands that did something
useful---for example, that created a graph we'd like to use in a paper.
We'd like to be able to re-create the graph later if we need to, so we
want to save the commands in a file. Instead of typing them in again
(and potentially getting them wrong) we can do this:

\begin{VerbIn}
$ history | tail -4 > redo-figure-3.sh
\end{VerbIn}

The file \code{redo-figure-3.sh} now contains:

\begin{VerbFile}
297 goostats -r NENE01729B.txt stats-NENE01729B.txt
298 goodiff stats-NENE01729B.txt /data/validated/01729.txt > 01729-differences.txt
299 cut -d ',' -f 2-3 01729-differences.txt > 01729-time-series.txt
300 ygraph --format scatter --color bw --borders none 01729-time-series.txt figure-3.png
\end{VerbFile}

\noindent
After a moment's work in an editor to remove the serial numbers on the
commands, we have a completely accurate record of how we created that
figure.

\begin{swcbox}{Unnumbering}

Nelle could also use \code{colrm} (short for ``column removal'') to
remove the serial numbers on her previous commands. Its arguments are
the range of characters to strip from its input:

\begin{VerbIn}
$ history | tail -5
\end{VerbIn}

% [c]
\begin{VerbOut}
  173  cd /tmp
  174  ls
  175  mkdir bakup
  176  mv bakup backup
  177  history | tail -5
\end{VerbOut}

% [c]
\begin{VerbIn}
$ history | tail -5 | colrm 1 7
\end{VerbIn}

% [c]
\begin{VerbOut}
cd /tmp
ls
mkdir bakup
mv bakup backup
history | tail -5
history | tail -5 | colrm 1 7
\end{VerbOut}

\end{swcbox}

In practice, most people develop shell scripts by running commands at
the shell prompt a few times to make sure they're doing the right thing,
then saving them in a file for re-use. This style of work allows people
to recycle what they discover about their data and their workflow with
one call to \code{history} and a bit of editing to clean up the output
and save it as a shell script.

\subsection*{Nelle's Pipeline: Creating a Script}

An off-hand comment from her supervisor has made Nelle realize that she
should have provided a couple of extra arguments to \code{goostats}
when she processed her files. This might have been a disaster if she had
done all the analysis by hand, but thanks to for loops, it will only
take a couple of hours to re-do.

But experience has taught her that if something needs to be done twice,
it will probably need to be done a third or fourth time as well. She
runs the editor and writes the following:

\begin{VerbFile}
# Calculate reduced stats for data files at J = 100 c/bp.
for datafile in $*
do
    echo $datafile
    goostats -J 100 -r $datafile stats-$datafile
done
\end{VerbFile}

\noindent
(The arguments \code{-J 100} and \code{-r} are the ones her
supervisor said she should have used.) She saves this in a file called
\code{do-stats.sh} so that she can now re-do the first stage of her
analysis by typing:

\begin{VerbIn}
$ bash do-stats.sh *[AB].txt
\end{VerbIn}

\noindent
She can also do this:

\begin{VerbIn}
$ bash do-stats.sh *[AB].txt | wc -l
\end{VerbIn}

\noindent
so that the output is just the number of files processed rather than the
names of the files that were processed.

One thing to note about Nelle's script is that it lets the person
running it decide what files to process. She could have written it as:

\begin{VerbFile}
# Calculate reduced stats for  A and Site B data files at J = 100 c/bp.
for datafile in *[AB].txt
do
    echo $datafile
    goostats -J 100 -r $datafile stats-$datafile
done
\end{VerbFile}

The advantage is that this always selects the right files: she doesn't
have to remember to exclude the `Z' files. The disadvantage is that it
\emph{always} selects just those files---she can't run it on all files
(including the `Z' files), or on the `G' or `H' files her colleagues in
Antarctica are producing, without editing the script. If she wanted to
be more adventurous, she could modify her script to check for
command-line arguments, and use \code{*{[}AB{]}.txt} if none were
provided. Of course, this introduces another tradeoff between
flexibility and complexity.

\begin{keypoints}
\begin{swcitemize}
\item
  Save commands in files (usually called shell scripts) for re-use.
\item
  \code{bash filename} runs the commands saved in a file.
\item
  \code{\$*} refers to all of a shell script's command-line
  arguments.
\item
  \code{\$1}, \code{\$2}, etc., refer to specified command-line
  arguments.
\item
  Letting users decide what files to process is more flexible and more
  consistent with built-in Unix commands.
\end{swcitemize}
\end{keypoints}

\begin{challenge}
  Leah has several hundred data files, each of which is formatted like
  this:

\begin{VerbFile}
2013-11-05,deer,5
2013-11-05,rabbit,22
2013-11-05,raccoon,7
2013-11-06,rabbit,19
2013-11-06,deer,2
2013-11-06,fox,1
2013-11-07,rabbit,18
2013-11-07,bear,1
\end{VerbFile}

  \noindent
  Write a shell script called \code{species.sh} that takes any number
  of filenames as command-line arguments, and uses \code{cut},
  \code{sort}, and \code{uniq} to print a list of the unique species
  appearing in each of those files separately.
\end{challenge}

\begin{challenge}
  Write a shell script called \code{longest.sh} that takes the name of
  a directory and a filename extension as its arguments, and prints out
  the name of the file with the most lines in that directory with that
  extension. For example:

\begin{VerbIn}
$ bash longest.sh /tmp/data pdb
\end{VerbIn}

  \noindent
  would print the name of the \code{.pdb} file in \code{/tmp/data}
  that has the most lines.
\end{challenge}

\begin{challenge}
  If you run the command:

\begin{VerbIn}
$ history | tail -5 > recent.sh
\end{VerbIn}

  \noindent
  the last command in the file is the \code{history} command itself,
  i.e., the shell has added \code{history} to the command log before
  actually running it. In fact, the shell \emph{always} adds commands to
  the log before running them. Why do you think it does this?
\end{challenge}

\begin{challenge}
  Joel's \code{data} directory contains three files:
  \code{fructose.dat}, \code{glucose.dat}, and \code{sucrose.dat}.
  Explain what a script called \code{example.sh} would do when run as
  \code{bash example.sh *.dat} if it contained the following lines:

\begin{VerbFile}
# Script 1
echo *.*
\end{VerbFile}

\begin{VerbFile}
# Script 2
for filename in $1 $2 $3
do
    cat $filename
done
\end{VerbFile}

\begin{VerbFile}
# Script 3
echo $*.dat
\end{VerbFile}

\end{challenge}

\section{Finding Things}

\begin{objectives}
\begin{swcitemize}
\item
  Use \code{grep} to select lines from text files that match simple
  patterns.
\item
  Use \code{find} to find files whose names match simple patterns.
\item
  Use the output of one command as the command-line arguments to
  another command.
\item
  Explain what is meant by ``text'' and ``binary'' files, and why many
  common tools don't handle the latter well.
\end{swcitemize}
\end{objectives}

You can guess someone's age by how they talk about search: young people
use ``Google'' as a verb, while crusty old Unix programmers use
``grep''. The word is a contraction of ``global/regular
expression/print'', a common sequence of operations in early Unix text
editors. It is also the name of a very useful command-line program.

\code{grep} finds and prints lines in files that match a pattern. For
our examples, we will use a file that contains three haikus taken from a
1998 competition in \emph{Salon} magazine:

\begin{VerbIn}
$ cat haiku.txt
\end{VerbIn}

% [c]
\begin{VerbOut}
The Tao that is seen
Is not the true Tao, until
You bring fresh toner.

With searching comes loss
and the presence of absence:
"My Thesis" not found.

Yesterday it worked
Today it is not working
Software is like that.
\end{VerbOut}

\begin{swcbox}{Forever, or Five Years}

We haven't linked to the original haikus because they don't appear to be
on \emph{Salon}'s site any longer. As
\urlfoot{http://www.clir.org/pubs/archives/ensuring.pdf}{Jeff Rothenberg
said}, ``Digital information lasts forever---or five years, whichever
comes first.''

\end{swcbox}

Let's find lines that contain the word ``not'':

\begin{VerbIn}
$ grep not haiku.txt
\end{VerbIn}

% [c]
\begin{VerbOut}
Is not the true Tao, until
"My Thesis" not found
Today it is not working
\end{VerbOut}

\noindent
Here, \code{not} is the pattern we're searching for. It's pretty
simple: every alphanumeric character matches against itself. After the
pattern comes the name or names of the files we're searching in. The
output is the three lines in the file that contain the letters ``not''.

Let's try a different pattern: ``day''.

\begin{VerbIn}
$ grep day haiku.txt
\end{VerbIn}

% [c]
\begin{VerbOut}
Yesterday it worked
Today it is not working
\end{VerbOut}

This time, the output is lines containing the words ``Yesterday'' and
``Today'', which both have the letters ``day''. If we give \code{grep}
the \code{-w} flag, it restricts matches to word boundaries, so that
only lines with the word ``day'' will be printed:

\begin{VerbIn}
$ grep -w day haiku.txt
\end{VerbIn}

\noindent
In this case, there aren't any, so \code{grep}'s output is empty.

Another useful option is \code{-n}, which numbers the lines that
match:

\begin{VerbIn}
$ grep -n it haiku.txt
\end{VerbIn}

% [c]
\begin{VerbOut}
5:With searching comes loss
9:Yesterday it worked
10:Today it is not working
\end{VerbOut}

\noindent
Here, we can see that lines 5, 9, and 10 contain the letters ``it''.

We can combine flags as we do with other Unix commands. For example,
since \code{-i} makes matching case-insensitive and \code{-v}
inverts the match, using them both only prints lines that \emph{don't}
match the pattern in any mix of upper and lower case:

\begin{VerbIn}
$ grep -i -v the haiku.txt
\end{VerbIn}

% [c]
\begin{VerbOut}
You bring fresh toner.

With searching comes loss

Yesterday it worked
Today it is not working
Software is like that.
\end{VerbOut}

\code{grep} has lots of other options. To find out what they are, we
can type \code{man grep}. \code{man} is the Unix ``manual'' command:
it prints a description of a command and its options, and (if you're
lucky) provides a few examples of how to use it:

\begin{VerbIn}
$ man grep
\end{VerbIn}

\begin{small}
% [c]
\begin{VerbOut}
GREP(1)                                                                                              GREP(1)

NAME
grep, egrep, fgrep - print lines matching a pattern

SYNOPSIS
grep [OPTIONS] PATTERN [FILE...]
grep [OPTIONS] [-e PATTERN | -f FILE] [FILE...]

DESCRIPTION
grep  searches the named input FILEs (or standard input if no files are named, or if a single hyphen-
minus (-) is given as file name) for lines containing a match to the given PATTERN.  By default, grep
prints the matching lines.
...        ...        ...

OPTIONS
Generic Program Information
--help Print  a  usage  message  briefly summarizing these command-line options and the bug-reporting
address, then exit.

-V, --version
Print the version number of grep to the standard output stream.  This version number should be
included in all bug reports (see below).

Matcher Selection
-E, --extended-regexp
Interpret  PATTERN  as  an  extended regular expression (ERE, see below).  (-E is specified by
POSIX.)

-F, --fixed-strings
Interpret PATTERN as a list of fixed strings, separated by newlines, any of  which  is  to  be
matched.  (-F is specified by POSIX.)
...        ...        ...
\end{VerbOut}
\end{small}

\begin{swcbox}{Wildcards}

\code{grep}'s real power doesn't come from its options, though; it
comes from the fact that patterns can include wildcards. (The technical
name for these is \gl{regular expressions}{g:regular-expression},
which is what the ``re'' in ``grep'' stands for.) Regular expressions
are both complex and powerful; if you want to do complex searches,
please look at the lesson on \urlfoot{http://software-carpentry.org}{our
website}. As a taster, we can find lines that have an `o' in the second
position like this:

\begin{VerbIn}
$ grep -E '^.o' haiku.txt
\end{VerbIn}

% [c]
\begin{VerbOut}
You bring fresh toner.
Today it is not working
Software is like that.
\end{VerbOut}

We use the \code{-E} flag and put the pattern in quotes to prevent the
shell from trying to interpret it. (If the pattern contained a `*', for
example, the shell would try to expand it before running \code{grep}.)
The `\textbackslash{}\^{}' in the pattern anchors the match to the start
of the line. The `.' matches a single character (just like `?' in the
shell), while the `o' matches an actual `o'.

\end{swcbox}

While \code{grep} finds lines in files, the \code{find} command
finds files themselves. Again, it has a lot of options; to show how the
simplest ones work, we'll use the directory tree shown in \figref{f:find-tree}.

\swcgraphics{f:find-tree}{Directory Tree for \code{find} Examples}{novice/shell/img/find-file-tree.pdf}{0.75}

Nelle's home directory contains one file called \code{notes.txt} and
four subdirectories: \code{thesis} (which is sadly empty),
\code{data} (which contains two files \code{one.txt} and
\code{two.txt}), a \code{tools} directory that contains the programs
\code{format} and \code{stats}, and an empty subdirectory called
\code{old}.

For our first command, let's run \code{find . -type d -print}. As
always, the \code{.} on its own means the current working directory,
which is where we want our search to start; \code{-type d} means
``things that are directories'', and (unsurprisingly) \code{-print}
means ``print what's found''. Sure enough, \code{find}'s output is the
names of the five directories in our little tree (including \code{.}):

\begin{VerbIn}
$ find . -type d -print
\end{VerbIn}

% [c]
\begin{VerbOut}
./
./data
./thesis
./tools
./tools/old
\end{VerbOut}

If we change \code{-type d} to \code{-type f}, we get a listing of
all the files instead:

\begin{VerbIn}
$ find . -type f -print
\end{VerbIn}

% [c]
\begin{VerbOut}
./data/one.txt
./data/two.txt
./notes.txt
./tools/format
./tools/stats
\end{VerbOut}

\code{find} automatically goes into subdirectories, their
subdirectories, and so on to find everything that matches the pattern
we've given it. If we don't want it to, we can use \code{-maxdepth} to
restrict the depth of search:

\begin{VerbIn}
$ find . -maxdepth 1 -type f -print
\end{VerbIn}

% [c]
\begin{VerbOut}
./notes.txt
\end{VerbOut}

The opposite of \code{-maxdepth} is \code{-mindepth}, which tells
\code{find} to only report things that are at or below a certain
depth. \code{-mindepth 2} therefore finds all the files that are two
or more levels below us:

\begin{VerbIn}
$ find . -mindepth 2 -type f -print
\end{VerbIn}

% [c]
\begin{VerbOut}
./data/one.txt
./data/two.txt
./tools/format
./tools/stats
\end{VerbOut}

Another option is \code{-empty}, which restricts matches to empty
files and directories:

\begin{VerbIn}
$ find . -empty -print
\end{VerbIn}

% [c]
\begin{VerbOut}
./thesis
./tools/old
\end{VerbOut}

Now let's try matching by name:

\begin{VerbIn}
$ find . -name *.txt -print
\end{VerbIn}

% [c]
\begin{VerbOut}
./notes.txt
\end{VerbOut}

\noindent
We expected it to find all the text files, but it only prints out
\code{./notes.txt}. The problem is that the shell expands wildcard
characters like \code{*} \emph{before} commands run. Since
\code{*.txt} in the current directory expands to \code{notes.txt},
the command we actually ran was:

\begin{VerbIn}
$ find . -name notes.txt -print
\end{VerbIn}

\noindent
\code{find} did what we asked; we just asked for the wrong thing.

To get what we want, let's do what we did with \code{grep}: put
\code{*.txt} in single quotes to prevent the shell from expanding the
\code{*} wildcard. This way, \code{find} actually gets the pattern
\code{*.txt}, not the expanded filename \code{notes.txt}:

\begin{VerbIn}
$ find . -name '*.txt' -print
\end{VerbIn}

% [c]
\begin{VerbOut}
./data/one.txt
./data/two.txt
./notes.txt
\end{VerbOut}

\begin{swcbox}{Listing vs. Finding}

\code{ls} and \code{find} can be made to do similar things given the
right options, but under normal circumstances, \code{ls} lists
everything it can, while \code{find} searches for things with certain
properties and shows them.

\end{swcbox}

As we said earlier, the command line's power lies in combining tools.
We've seen how to do that with pipes; let's look at another technique.
As we just saw, \code{find . -name '*.txt' -print} gives us a list of
all text files in or below the current directory. How can we combine
that with \code{wc -l} to count the lines in all those files?

The simplest way is to put the \code{find} command inside
\code{\$()}:

\begin{VerbIn}
$ wc -l $(find . -name '*.txt' -print)
\end{VerbIn}

% [c]
\begin{VerbOut}
70  ./data/one.txt
420  ./data/two.txt
30  ./notes.txt
520  total
\end{VerbOut}

When the shell executes this command, the first thing it does is run
whatever is inside the \code{\$()}. It then replaces the \code{\$()}
expression with that command's output. Since the output of \code{find}
is the three filenames \code{./data/one.txt}, \code{./data/two.txt},
and \code{./notes.txt}, the shell constructs the command:

\begin{VerbIn}
$ wc -l ./data/one.txt ./data/two.txt ./notes.txt
\end{VerbIn}

\noindent
which is what we wanted. This expansion is exactly what the shell does
when it expands wildcards like \code{*} and \code{?}, but lets us
use any command we want as our own ``wildcard''.

It's very common to use \code{find} and \code{grep} together. The
first finds files that match a pattern; the second looks for lines
inside those files that match another pattern. Here, for example, we can
find PDB files that contain iron atoms by looking for the string ``FE''
in all the \code{.pdb} files below the current directory:

\begin{VerbIn}
$ grep FE $(find . -name '*.pdb' -print)
\end{VerbIn}

% [c]
\begin{VerbOut}
./human/heme.pdb:ATOM  25  FE  1  -0.924  0.535  -0.518
\end{VerbOut}

\begin{swcbox}{Binary Files}

We have focused exclusively on finding things in text files. What if
your data is stored as images, in databases, or in some other format?
One option would be to extend tools like \code{grep} to handle those
formats. This hasn't happened, and probably won't, because there are too
many formats to support.

The second option is to convert the data to text, or extract the
text-ish bits from the data. This is probably the most common approach,
since it only requires people to build one tool per data format (to
extract information). On the one hand, it makes simple things easy to
do. On the negative side, complex things are usually impossible. For
example, it's easy enough to write a program that will extract X and Y
dimensions from image files for \code{grep} to play with, but how
would you write something to find values in a spreadsheet whose cells
contained formulas?

The third choice is to recognize that the shell and text processing have
their limits, and to use a programming language such as Python instead.
When the time comes to do this, don't be too hard on the shell: many
modern programming languages, Python included, have borrowed a lot of
ideas from it, and imitation is also the sincerest form of praise.

\end{swcbox}

\section{Conclusion}

The Unix shell is older than most of the people who use it. It has
survived so long because it is one of the most productive programming
environments ever created---maybe even \emph{the} most productive. Its
syntax may be cryptic, but people who have mastered it can experiment
with different commands interactively, then use what they have learned
to automate their work. Graphical user interfaces may be better at the
first, but the shell is still unbeaten at the second. And as Alfred
North Whitehead wrote in 1911, ``Civilization advances by extending the
number of important operations which we can perform without thinking
about them.''

\begin{keypoints}
\begin{swcitemize}
\item
  Use \code{find} to find files and directories, and \code{grep} to
  find text patterns in files.
\item
  \code{\$(command)} inserts a command's output in place.
\item
  \code{man command} displays the manual page for a given command.
\end{swcitemize}
\end{keypoints}

\begin{challenge}
  Write a short explanatory comment for the following shell script:

\begin{VerbIn}
find . -name '*.dat' -print | wc -l | sort -n
\end{VerbIn}
\end{challenge}

\begin{challenge}
  The \code{-v} flag to \code{grep} inverts pattern matching, so
  that only lines which do \emph{not} match the pattern are printed.
  Given that, which of the following commands will find all files in
  \code{/data} whose names end in \code{ose.dat} (e.g.,
  \code{sucrose.dat} or \code{maltose.dat}), but do \emph{not}
  contain the word \code{temp}?

  \begin{swcenumerate}
  \item
    \code{find /data -name '*.dat' -print \textbar{} grep ose \textbar{} grep -v temp}
  \item
    \code{find /data -name ose.dat -print \textbar{} grep -v temp}
  \item
    \code{grep -v temp \$(find /data -name '*ose.dat' -print)}
  \item
    None of the above.
  \end{swcenumerate}
\end{challenge}
