\chapter{Glossary}\label{s:gloss}

\gterm{absolute error}:
The absolute value of the difference between a mathematical value
and its finite approximation in a computer.

\gterm{absolute path}:
A \gl{path}{g:path} that refers to a
particular location in a file system. Absolute paths are usually written
with respect to the file system's \gl{root
directory}{g:root-directory}, and begin with either ``/'' (on Unix) or
``\textbackslash{}'' (on Microsoft Windows). See also:
\gl{relative path}{g:relative-path}.

\gterm{access control list} (ACL):
A list of permissions attached to a file or directory
that specifies who can do what with it.

\gterm{additive color model}:
A way to represent colors as the sum of
contributions from primary colors such as \gl{red, green,
and blue}{g:rgb}.

\gterm{aggregation function}:
A function such as \code{sum} or
\code{max} that combines many values to produce a single result.

\gterm{alias} (a library):
To give a \gl{library}{g:library} a
nickname while importing it.

\gterm{assertion}:
An expression which is supposed to be true at a
particular point in a program. Programmers typically put assertions in
their code to check for errors; if the assertion fails (i.e., if the
expression evaluates as false), the program halts and produces an error
message. See also: \gl{invariant}{g:invariant},
\gl{precondition}{g:precondition},
\gl{postcondition}{g:postcondition}.

\gterm{assignment}:
To give a value a name by associating a variable
with it.

\gterm{atomic value}:
A value that cannot be decomposed into smaller
pieces. For example, the number 12 is usually considered atomic (unless
we are teaching addition to school children, in which case we might
decompose it into tens and ones).

\gterm{branch}:
A "parallel universe" in a \gl{version control}{g:version-control} \gl{repository}{g:repository}.
Programmers typically use branches to isolate different sets of changes from one another during development
so that they can concentrate on one problem at a time.
See also: \gl{merge}{g:repository-merge}.

\gterm{call stack}:
A data structure inside a running program that
keeps track of active function calls. Each call's variables are stored
in a \gl{stack frame}{g:stack-frame}; a new stack frame is put on
top of the stack for each call, and discarded when the call is finished.

\gterm{cascading delete}:
The practice of automatically deleting things
in a database that depend on a record when that record is deleted. See
also: \gl{referential integrity}{g:referential-integrity}.

\gterm{case insensitive}:
Treating text as if upper and lower case
characters were the same. See also: \gl{case
sensitive}{g:case-sensitive}.
 
\gterm{catch} (an exception):
To handle an \gl{exception}{g:exception} that has been \gl{raised}{g:raise-exception}
somewhere else in a program.

\gterm{change set}:
A group of changes to one or more files that are
\gl{committed}{g:commit} to a \gl{version
control}{g:version-control} \gl{repository}{g:repository} in a single operation.

\gterm{clone} (a repository):
To make a local copy of a \gl{version control}{g:version-control} \gl{repository}{g:repository}.
See also: \gl{fork}{g:repository-fork}.

\gterm{code review}:
A systematic peer review of a piece of software,
or of changes to a piece of software.
Peer review is often conducted on \gl{pull requests}{g:pull-request}
before they are \gl{merged}{g:repository-merge} into a \gl{repository}{g:repository}.

\gterm{comma-separated values} (CSV):
A common textual representation
for tables in which the values in each row are separated by commas.

\gterm{command-line interface} (CLI):
An interface based on typing
commands, usually at a \gl{REPL}{g:repl}. See also:
\gl{graphical user interface}{g:gui}.

\gterm{comment}:
A remark in a program that is intended to help human
readers understand what is going on, but is ignored by the computer.
Comments in Python, R, and the Unix shell start with a \code{\#}
character and run to the end of the line; comments in SQL start with
\code{-{}-}, and other languages have other conventions.

\gterm{conditional statement}:
A statement in a program that might or
might not be executed depending on whether a test is true or false.

\gterm{conflict}:
A change made by one user of a
\gl{version control system}{g:version-control} that is
incompatible with changes made by other users. Helping users
\gl{resolve}{g:resolve} conflicts is one of version control's
major tasks.

\gterm{cross product}:
A pairing of all elements of one set with all
elements of another.

\gterm{current working directory}:
The directory that
\gl{relative paths}{g:relative-path} are calculated from;
equivalently, the place where files referenced by name only are searched
for. Every \gl{process}{g:process} has a current working
directory. The current working directory is usually referred to using
the shorthand notation \code{.} (pronounced ``dot'').

\gterm{cursor}:
A pointer into a database that keeps track of
outstanding operations.

\gterm{data type}:
A kind of data value, such as
\gl{integer}{g:integer} or \gl{character string}{g:string}.

\gterm{database manager}:
A program that manages a
\gl{relational database}{g:relational-database}.

\gterm{default parameter value}:
A value to use for a parameter if
nothing is specified explicitly.

\gterm{defensive programming}:
The practice of writing programs that
check their own operation to catch errors as early as possible.

\gterm{delimiter}:
A character or characters used to separate
individual values, such as the commas between columns in a
\gl{CSV}{g:csv} file.

\gterm{disk block}:
The smallest unit of storage that can be allocated
on a computer disk. Disk blocks are typically 512 bytes in size.

\gterm{docstring}:
Short for ``documentation string'', this refers to
textual documentation embedded in Python programs. Unlike comments,
docstrings are preserved in the running program and can be examined in
interactive sessions.

\gterm{documentation}:
Human-language text written to explain what
software does, how it works, or how to use it.

\gterm{dotted notation}:
A two-part notation used in many programming
languages in which \code{thing.component} refers to the
\code{component} belonging to \code{thing}.

\gterm{empty string}:
A character string containing no characters,
often thought of as the ``zero'' of text.

\gterm{encapsulation}:
The practice of hiding something's
implementation details so that the rest of a program can worry about
\emph{what} it does rather than \emph{how} it does it.

\gterm{exception}:
An event that disrupts the normal or expected execution of a program.
Most modern languages record information about what went wrong
in a piece of data (also called an exception).
See also: \gl{catch}{g:catch-exception}, \gl{raise}{g:raise-exception}.

\gterm{field} (of a database):
A set of data values of a particular
type, one for each \gl{record}{g:record-database} in a
\gl{table}{g:table-database}.

\gterm{filename extension}:
The portion of a file's name that comes
after the final ``.'' character. By convention this identifies the
file's type: \code{.txt} means ``text file'', \code{.png} means
``Portable Network Graphics file'', and so on. These conventions are not
enforced by most operating systems: it is perfectly possible to name an
MP3 sound file \code{homepage.html}. Since many applications use
filename extensions to identify the \gl{MIME type}{g:mime-type} of
the file, misnaming files may cause those applications to fail.

\gterm{filesystem}:
A set of files, directories, and I/O devices (such
as keyboards and screens). A filesystem may be spread across many
physical devices, or many filesystems may be stored on a single physical
device; the \gl{operating system}{g:operating-system} manages
access.

\gterm{filter}:
A program that transforms a stream of data. Many Unix
command-line tools are written as filters: they read data from
\gl{standard input}{g:standard-input}, process it, and write the
result to \gl{standard output}{g:standard-output}.

\gterm{flag}:
A terse way to specify an option or setting to a
command-line program. By convention Unix applications use a dash
followed by a single letter, such as \code{-v}, or two dashes followed
by a word, such as \code{-{}-verbose}, while DOS applications use a
slash, such as \code{/V}. Depending on the application, a flag may be
followed by a single argument, as in \code{-o /tmp/output.txt}.

\gterm{floating point number}:
A number containing a fractional
part and an exponent. See also: \gl{integer}{g:integer}.

\gterm{for loop}:
A loop that is executed once for each value in some
kind of set, list, or range. See also: \gl{while
loop}{g:while-loop}.

\gterm{foreign key}:
One or more values in a
\gl{database table}{g:table-database} that identify a
\gl{records}{g:record-database} in another table.

\gterm{fork}:
To \gl{clone}{g:repository-clone} a \gl{version control}{g:version-control} \gl{repository}{g:repository}
on a server.

\gterm{function call}:
A use of a function in another piece of
software.

\gterm{function body}:
The statements that are executed inside a
function.

\gterm{function composition}:
The immediate application of one function
to the result of another, such as \code{f(g(x))}.

\gterm{graphical user interface} (GUI):
A graphical user interface,
usually controlled by using a mouse. See also:
\gl{command-line interface}{g:cli}.

\gterm{home directory}:
The default directory associated with an
account on a computer system. By convention, all of a user's files are
stored in or below her home directory.

\gterm{HTTP}:
The Hypertext Transfer \gl{Protocol}{g:protocol} used for sharing web pages and other data
on the World Wide Web.

\gterm{immutable}:
Unchangeable. The value of immutable data cannot be
altered after it has been created. See also:
\gl{mutable}{g:mutable}.

\gterm{import}:
To load a \gl{library}{g:library} into a program.

\gterm{in-place operator}:
An operator such as \code{+=} that
provides a shorthand notation for the common case in which the variable
being assigned to is also an operand on the right hand side of the
assignment. For example, the statement \code{x += 3} means the same
thing as \code{x = x + 3}.

\gterm{index}:
A subscript that specifies the location of a single
value in a collection, such as a single pixel in an image.

\gterm{infective license}:
A license such as the
\urlfoot{http://opensource.org/licenses/GPL-3.0}{GPL} that compels people
who incorporate material into their own work to place similar sharing
requirements on it.

\gterm{inner loop}:
A loop that is inside another loop. See also:
\gl{outer loop}{g:outer-loop}.

\gterm{integer}:
A whole number, such as -12343. See also:
\gl{floating-point number}{g:float}.

\gterm{invariant}:
An expression whose value doesn't change during the
execution of a program, typically used in an
\gl{assertion}{g:assertion}. See also:
\gl{precondition}{g:precondition},
\gl{postcondition}{g:postcondition}.

\gterm{library}:
A family of code units (functions, classes, variables)
that implement a set of related tasks.

\gterm{loop body}:
The set of statements or commands that are repeated inside a \gl{for loop}{g:for-loop}
or \gl{while loop}{g:while-loop}.

\gterm{loop variable}:
The variable that keeps track of the progress of
the loop.

\gterm{member}:
A variable contained within an
\gl{object}{g:object}.

\gterm{merge} (a repository):
To reconcile two sets of change to a
\gl{repository}{g:repository}.

\gterm{method}:
A function which is tied to a particular
\gl{object}{g:object}. Each of an object's methods typically
implements one of the things it can do, or one of the questions it can
answer.

\gterm{mutable}:
Changeable. The value of mutable data can be updated
in place. See also: \gl{immutable}{g:immutable}.

\gterm{notional machine}:
An abstraction of a computer used to think
about what it can and will do.

\gterm{object}:
A particular ``chunk'' of data associated with specific
operations called \gl{methods}{g:method}.

\gterm{orthogonal}:
To have meanings or behaviors that are independent
of each other. If a set of concepts or tools are orthogonal, they can be
combined in any way.

\gterm{outer loop}:
A loop that contains another loop. See also:
\gl{inner loop}{g:inner-loop}.

\gterm{parameter}:
A value passed into a function, or a variable named
in the function's declaration that is used to hold such a value.

\gterm{parent directory}:
The directory that ``contains'' the one in
question. Every directory in a file system except the
\gl{root directory}{g:root-directory} has a parent. A directory's
parent is usually referred to using the shorthand notation \code{..}
(pronounced ``dot dot'').

\gterm{pipe}:
A connection from the output of one program to the input
of another. When two or more programs are connected in this way, they
are called a ``pipeline''.

\gterm{pipe and filter}:
A model of programming in which
\gl{filters}{g:filter} that process \gl{streams}{g:stream}
of data are connected end-to-end. The pipe and filter model is used
extensively in the Unix \gl{shell}{g:shell}.

\gterm{postcondition}:
A condition that a function (or other block of
code) guarantees is true once it has finished running. Postconditions
are often represented using \gl{assertions}{g:assertion}.

\gterm{precondition}:
A condition that must be true in order for a
function (or other block of code) to run correctly.

\gterm{prepared statement}:
A template for an \gl{SQL}{g:sql}
query in which some values can be filled in.

\gterm{primary key}:
One or more \gl{fields}{g:field-database} in
a \gl{database table}{g:table-database} whose values are
guaranteed to be unique for each \gl{record}{g:record-database},
i.e., whose values uniquely identify the entry.

\gterm{process}:
A running instance of a program, containing code,
variable values, open files and network connections, and so on.
Processes are the ``actors'' that the
\gl{operating system}{g:operating-system} manages; it typically
runs each process for a few milliseconds at a time to give the
impression that they are executing simultaneously.

\gterm{prompt}:
A character or characters display by a
\gl{REPL}{g:repl} to show that it is waiting for its next command.

\gterm{protocol}:
A set of rules that define how one computer communicates with another.
Common protocols on the Internet include \gl{HTTP}{g:http} and \gl{SSH}{g:ssh}.
 
\gterm{pull request}:
A set of changes created in one \gl{version control}{g:version-control} \gl{repository}{g:repository}
that is being offered to another for \gl{merging}{g:repository-merge}.

\gterm{query}:
A database operation that reads values but does not
modify anything. Queries are expressed in a special-purpose language
called \gl{SQL}{g:sql}.

\gterm{quoting} (in the shell):
Using quotation marks of various kinds
to prevent the shell from interpreting special characters. For example,
to pass the string \code{*.txt} to a program, it is usually necessary
to write it as \code{'*.txt'} (with single quotes) so that the shell
will not try to expand the \code{*} wildcard.
 
\gterm{raise} (an exception):
To explicitly signal that an \gl{exception}{g:exception} has occured in a program.
See also: \gl{catch}{g:catch-exception}.

\gterm{read-eval-print loop} (REPL):
A \gl{command-line
interface}{g:cli} that reads a command from the user, executes it, prints the
result, and waits for another command.

\gterm{record} (in a database):
A set of related values making up a
single entry in a \gl{database table}{g:table-database}, typically
shown as a row. See also: \gl{field}{g:field-database}.

\gterm{redirect}:
To send a command's output to a file rather than to
the screen or another command, or equivalently to read a command's input
from a file.

\gterm{referential integrity}:
The internal consistency of values in a
database. If an entry in one table contains a
\gl{foreign key}{g:foreign-key}, but the corresponding
\gl{records}{g:record-database} don't exist, referential integrity
has been violated.

\gterm{regression}:
To re-introduce a bug that was once fixed.

\gterm{regular expressions} (RE):
A pattern that specifies a set of
character strings. REs are most often used to find sequences of
characters in strings.

\gterm{relational database}:
A collection of data organized into
\gl{tables}{g:table-database}.
 
\gterm{relative error}:
The ratio of the \gl{absolute error}{g:absolute-error} in an approximation of a value
to the value being approximated.

\gterm{relative path}:
A \gl{path}{g:path} that specifies the
location of a file or directory with respect to the
\gl{current working directory}{g:current-working-directory}. Any
path that does not begin with a separator character (``/'' or
``\textbackslash{}'') is a relative path. See also:
\gl{absolute path}{g:absolute-path}.
 
\gterm{remote login}:
To connect to a computer over a network,
e.g., to run a \gl{shell}{g:shell} on it.
See also: \gl{SSH}{g:ssh}.

\gterm{remote repository}:
A version control
\gl{repository}{g:repository} other than the current one that the
current one is somehow connected to or mirroring.

\gterm{repository}:
A storage area where a
\gl{version control}{g:version-control} system stores old
\gl{revisions}{g:revision} of files and information about who
changed what, when.

\gterm{resolve}:
To eliminate the \gl{conflicts}{g:conflict}
between two or more incompatible changes to a file or set of files being
managed by a \gl{version control}{g:version-control} system.

\gterm{return statement}:
A statement that causes a function to stop
executing and return a value to its caller immediately.

\gterm{revision}:
A recorded state of a
\gl{version control}{g:version-control}
\gl{repository}{g:repository}.

\gterm{RGB}:
An \gl{additive model}{g:additive-color-model} that represents colors
as combinations of red, green, and blue. Each color's value is typically
in the range 0..255 (i.e., a one-byte integer).

\gterm{root directory}:
The top-most directory in a
\gl{filesystem}{g:filesystem}. Its name is ``/'' on Unix
(including Linux and Mac OS X) and ``\textbackslash{}'' on Microsoft
Windows.
 
\gterm{search path}:
The list of directories in which the \gl{shell}{g:shell} searches for programs when they are run.

\gterm{sentinel value}:
A value in a collection that has a special
meaning, such as 999 to mean ``age unknown''.

\gterm{sequence}:
An ordered collection of values whose elements can be
specified with a single integer index, such as a vector.

\gterm{shape} (of an array):
An array's dimensions, represented as a
vector. For example, a 5${\times}$3 array's shape is \code{(5,3)}.

\gterm{shell}:
A \gl{command-line interface}{g:cli} such as Bash
(the Bourne-Again Shell) or the Microsoft Windows DOS shell that allows
a user to interact with the \gl{operating
system}{g:operating-system}.

\gterm{shell script}:
A set of \gl{shell}{g:shell} commands
stored in a file for re-use. A shell script is a program executed by the
shell; the name ``script'' is used for historical reasons.
 
\gterm{sign and magnitude}:
A scheme for representing numbers in which one bit indicates the sign (positive or negative)
and the other bits store the number's absolute value.
See also: \gl{two's complement}{g:twos-complement}.

\gterm{silent failure}:
Failing without producing any warning messages.
Silent failures are hard to detect and debug.

\gterm{slice}:
A regular subsequence of a larger sequence, such as the
first five elements or every second element.

\gterm{SQL} (Structured Query Language):
A special-purpose language for
describing operations on \gl{relational
databases}{g:relational-database}.

\gterm{SQL injection attack}:
An attack on a program in which the
user's input contains malicious SQL statements. If this text is copied
directly into an SQL statement, it will be executed in the database.

\gterm{SSH}:
The Secure Shell \gl{protocol}{g:protocol} used for secure communication between computers.
SSH is often used for \gl{remote login}{g:remote-login} between computers.

\gterm{SSH key}:
A digital key that identifies one computer or user to another.

\gterm{stack frame}:
A data structure that provides storage for a
function's local variables. Each time a function is called, a new stack
frame is created and put on the top of the \gl{call
stack}{g:call-stack}. When the function returns, the stack frame is discarded.

\gterm{standard input} (stdin):
A process's default input stream. In
interactive command-line applications, it is typically connected to the
keyboard;; in a \gl{pipe}{g:pipe}, it receives data from the
\gl{standard output}{g:standard-output} of the preceding process.

\gterm{standard output} (stdout):
A process's default output stream. In
interactive command-line applications, data sent to standard output is
displayed on the screen; in a \gl{pipe}{g:pipe}, it is passed to
the \gl{standard input}{g:standard-input} of the next process.

\gterm{stride}:
The offset between successive elements of a
\gl{slice}{g:slice}.

\gterm{string}:
Short for ``character string'', a
\gl{sequence}{g:sequence} of zero or more characters.

\gterm{sub-directory}:
A directory contained within another directory.

\gterm{tab completion}:
A feature provided by many interactive systems
in which pressing the Tab key triggers automatic completion of the
current word or command.

\gterm{table} (in a database):
A set of data in a
\gl{relational database}{g:relational-database} organized into a
set of \gl{records}{g:record-database}, each having the same named
\gl{fields}{g:field-database}.

\gterm{test oracle}:
A program, device, data set, or human being
against which the results of a test can be compared.

\gterm{test-driven development} (TDD):
The practice of writing unit
tests \emph{before} writing the code they test.
 
\gterm{timestamp}:
A record of when a particular event occurred.

\gterm{tuple}:
An \gl{immutable}{g:immutable}
\gl{sequence}{g:sequence} of values.
 
\gterm{two's complement}:
A scheme for representing numbers which wraps around like an odometer
so that 111...111 represents -1.
See also: \gl{sign and magnitude}{g:sign-and-magnitude}.
 
\gterm{user group}:
A set of users on a computer system.
 
\gterm{user group ID}:
A numerical ID that specifies a \gl{user group}{g:user-group}.
 
\gterm{user group name}:
A textual name for a \gl{user group}{g:user-group}.
 
\gterm{user ID}:
A numerical ID that specifies an individual user on a computer system.
See also: \gl{user name}{g:user-name}.
 
\gterm{user name}:
A textual name for a user on a computer system.
See also: \gl{user ID}{g:user-id}.

\gterm{variable}:
A name in a program that is associated with a value
or a collection of values.

\gterm{version control}:
A tool for managing changes to a set of files.
Each set of changes creates a new \gl{revision}{g:revision} of the
files; the version control system allows users to recover old revisions
reliably, and helps manage conflicting changes made by different users.
 
\gterm{while loop}:
A loop that keeps executing as long as some condition is true.
See also: \gl{for loop}{g:for-loop}.

\gterm{wildcard}:
A character used in pattern matching. In the Unix
shell, the wildcard ``*'' matches zero or more characters, so that
\code{*.txt} matches all files whose names end in \code{.txt}.
